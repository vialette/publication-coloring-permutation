\section{Introduction}
\label{section:Introduction}

A permutation $\pi^0$ is a merge of permutations 
$\pi^1, \pi^2, \dots, \pi^k$, 
if there exists a $k$-coloring of $\pi^0$ (\emph{i.e.}, assign one color
among $k$ available colors to each element of $\pi^0$) such that the
pattern induced by each color $i$, $1 \leq i \leq k$, is order-isomorphic
to $\pi^i$.
For example, $\pi^0 = 192854367$ is a merge of  
${\color{red}12}$, ${\color{teal}312}$ and ${\color{blue}4213})$
as shown by 
$\pi = 
\raisebox{2pt}{{\color{red}1}}
\raisebox{0pt}{{\color{teal}92}}
\raisebox{-2pt}{{\color{blue}8}}
\raisebox{2pt}{{\color{red}5}}
\raisebox{-2pt}{{\color{blue}43}}
\raisebox{0pt}{{\color{teal}6}}
\raisebox{-2pt}{{\color{blue}7}}$,
${\color{red}15} \simeq 12$,
${\color{teal}926} \simeq 312$ and
${\color{blue}8437} \simeq 4213$.
Given permutations $\pi^0$ and $\pi^1, \pi^2, \dots, \pi^k$,
\textsc{$k$-Merge Permutation} is the problem to decide whether $\pi$
is a merge of $\pi^1, \pi^2, \dots, \pi^k$.

Given a permutation $\pi$, deciding whether $\pi$ is the shuffle of 
two order-isomorphic patterns is \NP-complete \cite{DBLP:journals/tcs/GiraudoV18}.
In other words, given a permutation $\pi$, deciding whether 
there exists a permutations $\sigma$ such that $\pi$ is 
$(\sigma, \sigma)$-colorable is \NP-complete,
even if $\pi$ is $(213, 231)$-avoiding.


For fixed hereditary permutation classes $\mathcal{C}$ and $\mathcal{D}$,
the complexity of determining whether a given permutation $\pi^0$ is a merge of an element of
$\mathcal{C}$ with an element of $\mathcal{D}$ is considered in
\cite{DBLP:conf/esa/JelinekOV18}.
In particular, it shown that it is \NP-complete to recognize permutations that can \begin{enumerate}
  \item merged from two permutations avoiding a simple pattern of length at least $4$.
\end{enumerate}