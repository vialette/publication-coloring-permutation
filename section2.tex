\section{Definitions}
\label{section:Definitions}

\subsection*{\textbf{Permutation}}

For any non-negative integer $n$, we let $[n]$ stand for
the set $\{1, 2, \dots, n\}$.
% We follow the usual terminology on words~\cite{ChoffrutKarhumaki1997}.
% Let us recall here the most important ones. Let $u$ be a word. The
% length of $u$ is denoted by $|u|$. The {\em empty word}, the only word
% of null length, is denoted by $\epsilon$. For any $i \in [|u|]$, the
% $i$-th letter of $u$ is denoted by $u(i)$. If $I$ is a subset of
% $[|u|]$, $u_{|I}$ is the subword of $u$ consisting in the letters of $u$
% at the positions specified by the elements of $I$.
% A \emph{permutation} of size $n$ is a word $\pi$ of length $n$ on the alphabet $[n]$ such
% that each letter admits exactly one occurrence. The set of all
% permutations of size $n$ is denoted by $S_n$.

\begin{definition}[Direct sum and skew sum]
  Given a permutation $\pi$ of length $m$ and the permutation $\sigma$
  of length $n$, the \emph{skew sum} of $\pi$ and $\sigma$ is the permutation
  of length $m+n$ defined by
  \begin{align*}
    (\pi \ominus \sigma )(i)
    &=
    \begin{cases}
      \pi(i)+n & \text{for $1\leq i \leq m$},\\
      \sigma(i-m) & \text{for $m+1 \leq i\leq m+n$,}
    \end{cases}
    \intertext{and the \emph{direct sum} of $\pi$ and $\sigma$ is the
    permutation of length $m+n$ defined by}
    (\pi \oplus \sigma )(i)
    &=
    \begin{cases}
      \pi(i) & \text{for $1\leq i \leq m$},\\
      \sigma(i-m) + m & \text{for $m+1 \leq i\leq m+n$.}
    \end{cases}
  \end{align*}
\end{definition}

\begin{definition}[Lifting]
  Let $\pi = \pi_1 \pi_1 \dots \pi_n$ be a permutation of size $n$ and
  $k$ be a positive integer.
  The \emph{$k$-lifting} of $\pi$, denoted $\pi \;[k]$,
  is the permutation
  $(k+\pi_1) (k+\pi_2) \dots (k+\pi_n)$.
\end{definition}

\begin{definition}[Monotone]
	For any positive integer $k$,
  we let $\mathbf{\nearrow}_k$ stand for the
  \emph{increasing permutation} $1 2 \dots k$
  and $\mathbf{\searrow}_k$ stand for the \emph{decreasing permutation}
  $k (k-1) \dots 1$.
\end{definition}

\begin{definition}[Reduced form]
    If $\pi$ is a permutation on some set $X$,
    the \emph{reduced form} of $\pi$, denoted $\RED(\pi)$, is the permutation
    obtained from $\pi$ by replacing its $i$-th smallest entry
    with $i$.
    For instance, $\RED(31845) = 21534$.
\end{definition}

\begin{definition}[Order-isomorphism]
    Two permutations $\sigma_1$ and $\sigma_2$ are \emph{order-isomorphic},
    denoted $\sigma_1 \simeq \sigma_2$,
    if $\RED(\sigma_1) = \RED(\sigma_2)$.
\end{definition}

\begin{definition}[Pattern containment]
    A permutation $\sigma$ is said to be \emph{contained} in, or to be
    a \emph{subpermutation} of, another permutation $\pi$, denoted
    $\sigma \preceq \pi$, if $\pi$ has a (not necessarily contiguous)
    subsequence whose terms are order-isomorphic to $\sigma$ (we also
    say that $\pi$ \emph{admits an occurrence} of the \emph{pattern}
    $\sigma$).
    If such a subsequence does not exist, $\pi$ is said to \emph{avoid}
    $\sigma$.
\end{definition}

\begin{definition}[permutation coloring]
  Let $\pi$ be a permutation of $X$.
  A \emph{$k$-coloring} of $\pi$ is a mapping $f : X \to [k]$.
  For every $i \in [k]$, the \emph{pattern induced} by color $i$ in $\pi$
  is the permutation obtained from $\pi$ by deleting all elements but
  those colored with color $i$.
\end{definition}

Given permutations $\pi$ and $\sigma_1, \sigma_2, \dots, \sigma_k$,
the permutation $\pi$ is said to be
\emph{$(\sigma_1, \sigma_2, \ldots, \sigma_k)$-colorable}
if there exists a
$k$-coloring of $\pi$ such that, for every $1 \leq i \leq k$, the
pattern of $\pi$ induced by color $i$ is order-isomorphic to $\sigma_i$.
The \textsc{$k$-Permutation Coloring} problem is to decide the existence of
such a $k$-coloring of $\pi$.

\begin{example}
The permutation $\pi = 483125679$ is $(123, 3214, 12)$-colorable
as shown in
$
\raisebox{1.5pt}{$4$}
\raisebox{0pt}{$8$}
\raisebox{0pt}{$3$}
\raisebox{-1.5pt}{$1$}
\raisebox{0pt}{$2$}
\raisebox{-1.5pt}{$5$}
\raisebox{1.5pt}{$6$}
\raisebox{1.5pt}{$7$}
\raisebox{0pt}{$9$}
$
where
$467 \simeq 123$,
$8329 \simeq 3214$ and
$15 \simeq 12$.
\end{example}

A \emph{separable permutation} is a permutation that has a \emph{separating tree}:
a rooted binary tree in which the elements
of the permutation appear (in permutation order) at the leaves of the tree,
and in which the descendants of each tree node form a contiguous subset of
these elements.
Each interior node of the tree is either a positive node in which all
descendants of the left child are smaller than all descendants of the right node,
or a negative node in which all descendants of the left node are greater than all
descendants of the right node.
Separable permutations may be characterized by the forbidden permutation
patterns $2413$ and $3142$.
The separable permutations are enumerated by the Schröder numbers.

\subsection*{\textbf{Linear graphs}}

A \emph{graph} is an ordered pair $G = (V, E)$ comprising a set $V$ of
vertices together with a set $E$ of edges which are 2-element subsets of $V$.
The \emph{order} and the \emph{size} of the graph $G$ stand for
$|V|$ and $|E|$, respectively.
A \emph{linear graph} is a graph equipped with a total order $<_G$
(or simply $<$ if the name of the graph is clear in the context).
In case of linear graphs, we write an edge between vertices $i$ and $j$ of
$G$, $i <_G j$, as the pair $(i, j)$.
Two edges of a graph are \emph{disjoint} if they do not share a common
vertex and a \emph{matching} is a set $\mathcal{M} \subseteq E$ of
pairwise independent edges.
A \emph{perfect matching} is a matching which matches all vertices of the
graph.
A \emph{connected component} of a graph is a subgraph in which any
two vertices are connected to each other by paths, and which is connected
to no additional vertices in the supergraph.
Let $G = (V, E)$ be a graph and $S \subseteq V$ be any subset of
vertices of $G$. Then the \emph{induced subgraph} $G[S]$ is the graph whose
vertex set is $S$ and whose edge set consists of all of the edges in $E$
that have both endpoints in $S$.
The above definitions do apply to linear graphs.

For a linear graph $G = (V,E)$, we shall usually write the vertices as a
word $w_G$, where the order $<_G$ is understood to be the left-to-right order
in $w_G$,
so that $w_G = u_1 u_2 \dots u_n$ stands for
$V = \{u_1, u_2, \dots, u_n\}$ with
$u_1 <_G u_2 <_G \cdots <_G u_n$.
This naturally extends to subsets $V' \subseteq V$ of $G$:
if $V'$ and $V''$ are two disjoint subsets of $V$, we let $V' <_G V''$
stand for $v' <_G v''$ for every $v' \in V'$ and every $v'' \in V''$.

Two linear graphs $G$ and $H$ are isomorphic,
denoted by $G \simeq H$,
if there exists an one-to-one mapping
$\varphi : V_G \to V_H$ such that
(i)
$i <_G j$ if and only if $\varphi(i) <_H \varphi(j)$ and
(ii)
$(i, j) \in E_G$ if and only if $(\varphi(i), \varphi(j)) \in E_H$.
\todo{Do we really need this definition?}

Given linear graphs $G^0$, and $G^1, G^2, \dots, G^k$,
the linear graph $G^0$ is said to be
\emph{$(G^1, G^2, \ldots, G^k)$-colorable}
if there exists a $k$-coloring of the connected components of $H$ such that
the linear subgraph $G^0[i]$ induced by the subset of all vertices with the
same color $i$, $1 \leq i \leq k$, is isomorphic to $G^i$.
Notice that the definition does not require the linear graphs
$G^1, G^2, \ldots, G^k$ to be necessarily connected
and hence that distinct connected components of $G^0$ may receive the
same color in the $k$-vertex coloring.

Given linear graphs $G^0$ and $G^1, G^2, \ldots, G^k$,
the \textsc{$k$-Linear Graph Coloring} problem is to decide whether $G^0$
is $(G^1, G^2, \ldots, G^k)$ -colorable.
An illustration of this definition is given in
Figure~\ref{fig:example decomposition}.

\begin{figure}
  \centering
  \begin{tikzpicture}
    [
      scale=1.,
      vertex/.style = {circle, fill, minimum size=6pt,inner sep=0pt, outer sep=0pt},
    ]
    \draw[xstep=1,ystep=.5,black,ultra thin,black!5] (-0.5,-12.25) grid (12.5,3.25);

    % graph G^0
    \draw (-1,0) node {$G^0$};

    % vertices
    \foreach [evaluate=\x as \l using int(\x+1)] \x in {0,...,12}
      \node [vertex,label=below:$v^{0}_{\l}$] (VG0\x) at (\x,0) {};

    % black
    \draw [edge]
      (VG00) ..controls ($(VG00.north) + (0,3.75)$) and ($(VG011.north) + (0,3.75)$) .. node [above] {$e^{0}_{1}$} (VG011.north);
    \draw [edge]
      (VG09) ..controls ($(VG09.north) + (0,.75)$) and ($(VG011.north) + (0,.75)$) .. node [above] {$e^{0}_{11}$} (VG011.north);
    \draw [edge]
      (VG02) ..controls ($(VG02.north) + (0,.5)$) and ($(VG03.north) + (0,.5)$) .. node [above] {$e^{0}_{6}$} (VG03.north);
    \draw [edge, rounded corners]
      (VG03) ..controls ($(VG03.north) + (0,.75)$) and ($(VG05.north) + (0,.75)$) .. node [above] {$e^{0}_{7}$} (VG05.north);
    \draw [edge, rounded corners]
      (VG02) ..controls ($(VG02.north) + (0,1.75)$) and ($(VG05.north) + (0,1.75)$) .. node [above] {$e^{0}_{3}$} (VG05.north);
    % white
    \draw [edge]
      (VG01) ..controls ($(VG01.north) + (0,2.75)$) and ($(VG08.north) + (0,2.75)$) .. node [above] {$e^{0}_{2}$} (VG08.north);
    \draw [edge]
      (VG08) ..controls ($(VG08.north) + (0,.75)$) and ($(VG010.north) + (0,.75)$) .. node [above] {$e^{0}_{10}$} (VG010.north);
    % gray
    \draw [edge]
      (VG04) ..controls ($(VG04.north) + (0,.75)$) and ($(VG06.north) + (0,.75)$) .. node [above] {$e^{0}_{8}$} (VG06.north);
    \draw [edge]
      (VG04) ..controls ($(VG04.north) + (0,1.75)$) and ($(VG07.north) + (0,1.75)$) .. node [above] {$e^{0}_{4}$} (VG07.north);
    \draw [edge]
      (VG06) ..controls ($(VG06.north) + (0,.5)$) and ($(VG07.north) + (0,.5)$) .. node [above] {$e^{0}_{9}$} (VG07.north);
    \draw [edge]
      (VG07) ..controls ($(VG07.north) + (0,1.75)$) and ($(VG012.north) + (0,1.75)$) .. node [above] {$e^{0}_{5}$} (VG012.north);

    % graph G_1
    \begin{scope}[yshift=-5cm]
      \draw (-1,0) node {$G^1$};

      \foreach \x/\l in {0/1,2/2,3/3,5/4,9/5,11/6}
        \node [vertex,fill=black,label=below:$v^{1}_{\l}$] (VG1\x) at (\x,0) {};

    \draw [edge]
      (VG10) ..controls ($(VG10.north) + (0,3.75)$) and ($(VG111.north) + (0,3.75)$) .. node [above] {$e^{1}_{1}$} (VG111.north);
    \draw [edge]
      (VG19) ..controls ($(VG19.north) + (0,.75)$) and ($(VG111.north) + (0,.75)$) .. node [above] {$e^{1}_{5}$} (VG111.north);
    \draw [edge]
      (VG12) ..controls ($(VG12.north) + (0,.5)$) and ($(VG13.north) + (0,.5)$) .. node [above] {$e^{1}_{3}$} (VG13.north);
    \draw [edge, rounded corners]
      (VG13) ..controls ($(VG13.north) + (0,.75)$) and ($(VG15.north) + (0,.75)$) .. node [above] {$e^{1}_{4}$} (VG15.north);
    \draw [edge, rounded corners]
      (VG12) ..controls ($(VG12.north) + (0,1.75)$) and ($(VG15.north) + (0,1.75)$) .. node [above] {$e^{1}_{2}$} (VG15.north);
    \end{scope}

    % graph G_2
    \begin{scope}[yshift=-8.5cm]
      % graph H
      \draw (-1,0) node {$G^2$};

      \foreach \x/\l in {1/1,8/2,10/3}
        \node [vertex,fill=black,label=below:$v^{2}_{\l}$] (VG2\x) at (\x,0) {};

    \draw [edge]
        (VG21) ..controls ($(VG21.north) + (0,2.75)$) and ($(VG28.north) + (0,2.75)$) .. node [above] {$e^{2}_{1}$} (VG28.north);
    \draw [edge]
        (VG28) ..controls ($(VG28.north) + (0,.75)$) and ($(VG210.north) + (0,.75)$) .. node [above] {$e^{2}_{2}$} (VG210.north);
    \end{scope}

    % graph G_3
    \begin{scope}[yshift=-12cm]
      % graph H
      \draw (-1,0) node {$G^3$};

      \foreach \x/\l in {4/1,6/2,7/3,12/4}
        \node [vertex,fill=black,label=below:$v^{3}_{\l}$] (VG3\x) at (\x,0) {};

    \draw [edge]
      (VG34) ..controls ($(VG34.north) + (0,.75)$) and ($(VG36.north) + (0,.75)$) .. node [above] {$e^{3}_{3}$} (VG36.north);
    \draw [edge]
      (VG34) ..controls ($(VG34.north) + (0,1.75)$) and ($(VG37.north) + (0,1.75)$) .. node [above] {$e^{3}_{1}$} (VG37.north);
    \draw [edge]
      (VG36) ..controls ($(VG36.north) + (0,.5)$) and ($(VG37.north) + (0,.5)$) .. node [above] {$e^{3}_{4}$} (VG37.north);
    \draw [edge]
      (VG37) ..controls ($(VG37.north) + (0,1.75)$) and ($(VG312.north) + (0,1.75)$) .. node [above] {$e^{3}_{2}$} (VG312.north);
    \end{scope}

  \end{tikzpicture}
  \caption{\label{fig:3-Linear-Graph Coloring}%
  A $(G^0 \,\mid\, G^1, G^2, G^3)$ instance of \textsc{$3$-Linear-Graph Coloring} together with an alignment
  describing a solution.}
\end{figure}

