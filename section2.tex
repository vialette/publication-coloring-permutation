\section{Definitions}
\label{section:Definitions}

For any non-negative integer $n$, we let $[n]$ stand for
the set $\{1, 2, \dots, n\}$.

\subsection*{\textbf{Permutation}}

\begin{definition}[Direct sum and skew sum]
  Given a permutation $\pi$ of length $m$ and the permutation $\sigma$
  of length $n$, the \emph{skew sum} of $\pi$ and $\sigma$ is the permutation
  of length $m+n$ defined by
  \begin{align*}
    (\pi \ominus \sigma )(i)
    &=
    \begin{cases}
      \pi(i)+n & \text{for $1\leq i \leq m$},\\
      \sigma(i-m) & \text{for $m+1 \leq i\leq m+n$,}
    \end{cases}
    \intertext{and the \emph{direct sum} of $\pi$ and $\sigma$ is the
    permutation of length $m+n$ defined by}
    (\pi \oplus \sigma )(i)
    &=
    \begin{cases}
      \pi(i) & \text{for $1\leq i \leq m$},\\
      \sigma(i-m) + m & \text{for $m+1 \leq i\leq m+n$.}
    \end{cases}
  \end{align*}
\end{definition}

\begin{definition}[Lifting]
  Let $\pi = \pi_1 \pi_1 \dots \pi_n$ be a permutation of size $n$ and
  $k$ be a positive integer.
  The \emph{$k$-lifting} of $\pi$, denoted $\pi \;[k]$,
  is the permutation
  $(k+\pi_1) (k+\pi_2) \dots (k+\pi_n)$.
\end{definition}

\begin{definition}[Monotone]
	For any positive integer $k$,
  we let $\mathbf{\nearrow}_k$ stand for the
  \emph{increasing permutation} $1 2 \dots k$
  and $\mathbf{\searrow}_k$ stand for the \emph{decreasing permutation}
  $k (k-1) \dots 1$.
\end{definition}

\begin{definition}[Reduced form]
    If $\pi$ is a permutation on some set $X$,
    the \emph{reduced form} of $\pi$, denoted $\RED(\pi)$, is the permutation
    obtained from $\pi$ by replacing its $i$-th smallest entry
    with $i$.
    For instance, $\RED(31845) = 21534$.
\end{definition}

\begin{definition}[Order-isomorphism]
    Two permutations $\sigma_1$ and $\sigma_2$ are \emph{order-isomorphic},
    denoted $\sigma_1 \simeq \sigma_2$,
    if $\RED(\sigma_1) = \RED(\sigma_2)$.
\end{definition}

\begin{definition}[Pattern containment]
    A permutation $\sigma$ is said to be \emph{contained} in, or to be
    a \emph{subpermutation} of, another permutation $\pi$, denoted
    $\sigma \preceq \pi$, if $\pi$ has a (not necessarily contiguous)
    subsequence whose terms are order-isomorphic to $\sigma$ (we also
    say that $\pi$ \emph{admits an occurrence} of the \emph{pattern}
    $\sigma$).
    If such a subsequence does not exist, $\pi$ is said to \emph{avoid}
    $\sigma$.
\end{definition}

\begin{definition}[permutation coloring]
  Given permutations $\pi$ and $\sigma_1, \sigma_2, \dots, \sigma_k$,
  the permutation $\pi$ is said to be
  \emph{$(\sigma_1, \sigma_2, \ldots, \sigma_k)$-colorable}
  if there exists a
  $k$-coloring of $\pi$ such that, for every $1 \leq i \leq k$, the
  pattern of $\pi$ induced by color $i$ is order-isomorphic to $\sigma_i$.
  The \textsc{$k$-Permutation Coloring} problem is to decide the existence of
  such a $k$-coloring of $\pi$.
\end{definition}

\begin{example}
The permutation $\pi = 483125679$ is $(123, 3214, 12)$-colorable
as shown in
$
\raisebox{1.5pt}{$4$}
\raisebox{0pt}{$8$}
\raisebox{0pt}{$3$}
\raisebox{-1.5pt}{$1$}
\raisebox{0pt}{$2$}
\raisebox{-1.5pt}{$5$}
\raisebox{1.5pt}{$6$}
\raisebox{1.5pt}{$7$}
\raisebox{0pt}{$9$}
$
where
$467 \simeq 123$,
$8329 \simeq 3214$ and
$15 \simeq 12$.
\end{example}

Given permutations $\pi$ and $\sigma^1, \sigma^2, \ldots, \sigma^k$,
the \textsc{$k$-Permutation Coloring} problem is to decide whether $\pi0$
is $(\sigma^1, \sigma^2, \ldots, \sigma^k)$ -colorable.

A \emph{separable permutation} is a permutation that has a \emph{separating tree}:
a rooted binary tree in which the elements
of the permutation appear (in permutation order) at the leaves of the tree,
and in which the descendants of each tree node form a contiguous subset of
these elements.
Each interior node of the tree is either a positive node in which all
descendants of the left child are smaller than all descendants of the right node,
or a negative node in which all descendants of the left node are greater than all
descendants of the right node.
Separable permutations may be characterized by the forbidden permutation
patterns $2413$ and $3142$.
The separable permutations are enumerated by the Schröder numbers.

\subsection*{\textbf{Matchings}}

A \emph{matching} $\mathcal{M}$ of $[2n]$ is a partition of $[2n]$ into size-$2$ parts,
and we refer to every size-$2$ part of $\mathcal{M}$ as an \emph{edge}.
Given linear matchings $\mathcal{M}^0$, and $\mathcal{M}^1, \mathcal{M}^2, \dots, \mathcal{M}^k$,
the matching $\mathcal{M}^0$ is said to be
\emph{$(\mathcal{M}^1, \mathcal{M}^2, \ldots, \mathcal{M}^k)$-colorable}
if there exists a $k$-coloring of the edges $\mathcal{M}^0$ such that
the sub-matching $\mathcal{M}^0[i]$ induced by the subset of all edges with the
same color $i$, $1 \leq i \leq k$, is isomorphic to $\mathcal{M}^i$.
An illustration of this definition is given in
Figure~\ref{fig:example 3-matching colorable}.

Given matchings $G\mathcal{M}^0$ and $\mathcal{M}^1, \mathcal{M}^2, \ldots, \mathcal{M}^k$,
the \textsc{$k$-Matching Coloring} problem is to decide whether $\mathcal{M}^0$
is $(\mathcal{M}^1, \mathcal{M}^2, \ldots, \mathcal{M}^k)$ -colorable.


\begin{figure}
  \centering
  \begin{tikzpicture}
    [
      scale=.65,
      vertex/.style = {circle, fill, minimum size=6pt,inner sep=0pt, outer sep=0pt},
      red vertex/.style = {vertex, ball color=red!80},
      green vertex/.style = {vertex, ball color=teal!80},
      blue vertex/.style = {vertex, ball color=blue!80},
      edge/.style = {ultra thick},
      red edge/.style = {edge, red},
      green edge/.style = {edge, teal},
      blue edge/.style = {edge, blue},
    ]
    %\draw[xstep=1,ystep=.5,black,ultra thin,black!5] (-0.5,-12.25) grid (12.5,3.25);

    % graph G^0
    \draw (-.5,0) node {$\mathcal{M}^0$};

    % red vertices
    \foreach \x in {3, 6, 8, 12, 14, 16}
      \node [red vertex,label=below:$\x$] (VG0\x) at (\x,0) {};
    
    % green edges
    \foreach \x in {7, 9, 10, 13, 15, 17}
      \node [green vertex,label=below:$\x$] (VG0\x) at (\x,0) {};
    
    % blue edges
    \foreach \x in {1, 2, 4, 5, 11, 18, 19, 20}
      \node [blue vertex,label=below:$\x$] (VG0\x) at (\x,0) {};

    % red edges
    \draw [red edge]
      (VG03) ..controls ($(VG03.north) + (0,2)$) and ($(VG08.north) + (0,2)$) .. (VG08.north);
    \draw [red edge]
      (VG06) ..controls ($(VG06.north) + (0,2)$) and ($(VG014.north) + (0,2)$) .. (VG014.north);
    \draw [red edge]
      (VG012) ..controls ($(VG012.north) + (0,1.5)$) and ($(VG016.north) + (0,1.5)$) .. (VG016.north);

    % green edges
    \draw [edge, white, ultra thick]
      (VG010) ..controls ($(VG010.north) + (0,1)$) and ($(VG013.north) + (0,1)$) .. (VG013.north);
    \draw [green edge]
      (VG010) ..controls ($(VG010.north) + (0,1)$) and ($(VG013.north) + (0,1)$) .. (VG013.north);
    \draw [green edge]
      (VG09) ..controls ($(VG09.north) + (0,2)$) and ($(VG015.north) + (0,2)$) .. (VG015.north);
    \draw [green edge]
      (VG07) ..controls ($(VG07.north) + (0,3)$) and ($(VG017.north) + (0,3)$) .. (VG017.north);

    % blue edges
    \draw [blue edge]
      (VG01) ..controls ($(VG01.north) + (0,2)$) and ($(VG05.north) + (0,2)$) .. (VG05.north);
    \draw [blue edge]
      (VG02) ..controls ($(VG02.north) + (0,1)$) and ($(VG04.north) + (0,1)$) .. (VG04.north);
    \draw [blue edge]
      (VG011) ..controls ($(VG011.north) + (0,2)$) and ($(VG019.north) + (0,2)$) .. (VG019.north);
    \draw [blue edge]
      (VG018) ..controls ($(VG018.north) + (0,1)$) and ($(VG020.north) + (0,1)$) .. (VG020.north);

    % graph G_1 (red)
    \begin{scope}[yshift=-3cm]
      \draw (-.5,0) node {$\mathcal{M}^1$};

      % red vertices
      \foreach \x in {1,...,6}
        \node [red vertex,label=below:$\x$] (VG1\x) at (\x,0) {};

      % red edges
      \draw [red edge]
        (VG11) ..controls ($(VG11.north) + (0,1)$) and ($(VG13.north) + (0,1)$) .. (VG13.north);
      \draw [red edge]
        (VG12) ..controls ($(VG12.north) + (0,1.5)$) and ($(VG15.north) + (0,1.5)$) .. (VG15.north);
      \draw [red edge]
        (VG14) ..controls ($(VG14.north) + (0,1)$) and ($(VG16.north) + (0,1)$) .. (VG16.north);
    \end{scope}

    % graph G_2 (green)
    \begin{scope}[yshift=-6cm]
      \draw (-.5,0) node {$\mathcal{M}^2$};

      % red vertices
      \foreach \x in {1,...,6}
        \node [green vertex,label=below:$\x$] (VG2\x) at (\x,0) {};

      % red edges
      \draw [green edge]
        (VG21) ..controls ($(VG21.north) + (0,2)$) and ($(VG26.north) + (0,2)$) .. (VG26.north);
      \draw [green edge]
        (VG22) ..controls ($(VG22.north) + (0,1.5)$) and ($(VG25.north) + (0,1.5)$) .. (VG25.north);
      \draw [green edge]
        (VG23) ..controls ($(VG23.north) + (0,1)$) and ($(VG24.north) + (0,1)$) .. (VG24.north);
    \end{scope}

    % graph G_3 (blue)
    \begin{scope}[yshift=-9cm]
      \draw (-.5,0) node {$\mathcal{M}^3$};

      % red vertices
      \foreach \x in {1,...,8}
        \node [blue vertex,label=below:$\x$] (VG3\x) at (\x,0) {};

      % red edges
      \draw [blue edge]
        (VG31) ..controls ($(VG31.north) + (0,1.5)$) and ($(VG34.north) + (0,1.5)$) .. (VG34.north);
      \draw [blue edge]
        (VG32) ..controls ($(VG32.north) + (0,1)$) and ($(VG33.north) + (0,1)$) .. (VG33.north);
      \draw [blue edge]
        (VG35) ..controls ($(VG35.north) + (0,1)$) and ($(VG37.north) + (0,1)$) .. (VG37.north);
      \draw [blue edge]
        (VG36) ..controls ($(VG36.north) + (0,1)$) and ($(VG38.north) + (0,1)$) .. (VG38.north);

    \end{scope}

  \end{tikzpicture}
  \caption{\label{fig:example 3-matching colorable}%
  A example instance of \textsc{$3$-Linear-Matching Coloring}.}
\end{figure}

