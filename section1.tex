\section{Introduction}
\label{section:Introduction}

A permutation $\pi^0 \in S(n)$ \emph{contains} another pemrutation $\pi^1 \in S(k)$,
denoted $\pi^1 \preceq \pi^0$, 
if $\pi^0$ contains a subsequence 
$\pi^0(i_1) \, \pi^0(i_2) \, \dots \, \pi^0(i_k)$ 
whose elements have the same relative order as the elements of $\pi^1$,
that is, for any $1 \leq x < y \leq k$,
$\pi^1(x) < \pi^1(y)$ if and only if
$\pi^(i_x) < \pi^0(i_y)$.
Such a subsequence is called an \emph{occurrence} of
$\pi^1$ in $\pi^0$.
If $\pi^0$ does not contain $\pi^1$ then we say that 
$\pi^1$ \emph{avoids} $\pi^1$.
Given $\pi^0 \in S(n)$ and $\pi^1 \in S(k)$,
\textsc{Permutation Pattern Matching} is the problem if 
$\pi^0$ contains $\pi^1$.
\textsc{Permutation Pattern Matching} is \NP-complete 
\cite{DBLP:journals/ipl/BoseBL98},
even if 
$\pi^0$ avoids $4321$ and $\pi^1$ avoids $321$
\cite{DBLP:conf/soda/JelinekK17}.
Guillemot and Marx have shown that \textsc{Permutation Pattern Matching} 
can be solved in time $n \, 2^{O(k^2 \log k)}$ 
\cite{DBLP:conf/soda/GuillemotM14}.

A permutation $\pi^0$ is a merge of permutations 
$\pi^1, \pi^2, \dots, \pi^k$, 
if there exists a $k$-coloring of $\pi^0$ (\emph{i.e.}, assign one color
among $k$ available colors to each element of $\pi^0$) such that the
pattern induced by each color $i$, $1 \leq i \leq k$, is order-isomorphic
to $\pi^i$.
For example, $\pi^0 = 192854367$ is a merge of  
${\color{red}12}$, ${\color{teal}312}$ and ${\color{blue}4213}$
as shown by 
$\pi = 
\raisebox{2pt}{{\color{red}1}}
\raisebox{0pt}{{\color{teal}92}}
\raisebox{-2pt}{{\color{blue}8}}
\raisebox{2pt}{{\color{red}5}}
\raisebox{-2pt}{{\color{blue}43}}
\raisebox{0pt}{{\color{teal}6}}
\raisebox{-2pt}{{\color{blue}7}}$,
${\color{red}15} \simeq 12$,
${\color{teal}926} \simeq 312$ and
${\color{blue}8437} \simeq 4213$.
Given permutations $\pi^0$ and $\pi^1, \pi^2, \dots, \pi^k$,
\textsc{$k$-Merge Permutation} is the problem to decide whether $\pi$
is a merge of $\pi^1, \pi^2, \dots, \pi^k$.


Given permutation $\pi^0$, deciding whether there exists some permutation 
$\pi^1$ such that $\pi^0$ is the merge of $\pi^1$ with $\pi^1$
is \NP-complete \cite{DBLP:journals/tcs/GiraudoV18},
even if $\pi$ is $(213, 231)$-avoiding
\cite{DBLP:journals/tcs/GiraudoV18,DBLP:journals/tcs/BulteauV20}.


For fixed hereditary permutation classes $\mathcal{C}$ and $\mathcal{D}$,
the complexity of determining whether a given permutation $\pi^0$ is a merge of an element of
$\mathcal{C}$ with an element of $\mathcal{D}$ is considered in
\cite{DBLP:conf/esa/JelinekOV18}.
In particular, it shown that it is \NP-complete to recognize permutations that can \begin{enumerate}
  \item merged from two permutations avoiding a simple pattern of length at least $4$.
\end{enumerate}