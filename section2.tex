\section{Definitions}
\label{section:Definitions}

For any non-negative integer $n$, we let $[n]$ stand for
the set $\{1, 2, \dots, n\}$.

\subsection*{\textbf{Permutation}}

\begin{definition}[Direct sum and skew sum]
  Given a permutation $\pi$ of length $m$ and the permutation $\sigma$
  of length $n$, the \emph{skew sum} of $\pi$ and $\sigma$ is the permutation
  of length $m+n$ defined by
  \begin{align*}
    (\pi \ominus \sigma )(i)
    &=
    \begin{cases}
      \pi(i)+n & \text{for $1\leq i \leq m$},\\
      \sigma(i-m) & \text{for $m+1 \leq i\leq m+n$,}
    \end{cases}
    \intertext{and the \emph{direct sum} of $\pi$ and $\sigma$ is the
    permutation of length $m+n$ defined by}
    (\pi \oplus \sigma )(i)
    &=
    \begin{cases}
      \pi(i) & \text{for $1\leq i \leq m$},\\
      \sigma(i-m) + m & \text{for $m+1 \leq i\leq m+n$.}
    \end{cases}
  \end{align*}
\end{definition}

\begin{definition}[Lifting]
  Let $\pi = \pi_1 \pi_1 \dots \pi_n$ be a permutation of size $n$ and
  $k$ be a positive integer.
  The \emph{$k$-lifting} of $\pi$, denoted $\pi \;[k]$,
  is the permutation
  $(k+\pi_1) (k+\pi_2) \dots (k+\pi_n)$.
\end{definition}

\begin{definition}[Monotone]
	For any positive integer $k$,
  we let $\mathbf{\nearrow}_k$ stand for the
  \emph{increasing permutation} $1 2 \dots k$
  and $\mathbf{\searrow}_k$ stand for the \emph{decreasing permutation}
  $k (k-1) \dots 1$.
\end{definition}

\begin{definition}[Reduced form]
    If $\pi$ is a permutation on some set $X$,
    the \emph{reduced form} of $\pi$, denoted $\RED(\pi)$, is the permutation
    obtained from $\pi$ by replacing its $i$-th smallest entry
    with $i$.
    For instance, $\RED(31845) = 21534$.
\end{definition}

\begin{definition}[Order-isomorphism]
    Two permutations $\sigma_1$ and $\sigma_2$ are \emph{order-isomorphic},
    denoted $\sigma_1 \simeq \sigma_2$,
    if $\RED(\sigma_1) = \RED(\sigma_2)$.
\end{definition}

\begin{definition}[Pattern containment]
    A permutation $\sigma$ is said to be \emph{contained} in, or to be
    a \emph{subpermutation} of, another permutation $\pi$, denoted
    $\sigma \preceq \pi$, if $\pi$ has a (not necessarily contiguous)
    subsequence whose terms are order-isomorphic to $\sigma$ (we also
    say that $\pi$ \emph{admits an occurrence} of the \emph{pattern}
    $\sigma$).
    If such a subsequence does not exist, $\pi$ is said to \emph{avoid}
    $\sigma$.
\end{definition}

\begin{definition}[permutation coloring]
  Given permutations $\pi$ and $\sigma_1, \sigma_2, \dots, \sigma_k$,
  the permutation $\pi$ is said to be
  \emph{$(\sigma_1, \sigma_2, \ldots, \sigma_k)$-colorable}
  if there exists a
  $k$-coloring of $\pi$ such that, for every $1 \leq i \leq k$, the
  pattern of $\pi$ induced by color $i$ is order-isomorphic to $\sigma_i$.
  The \textsc{$k$-Permutation Coloring} problem is to decide the existence of
  such a $k$-coloring of $\pi$.
\end{definition}

\begin{example}
The permutation $\pi = 483125679$ is $(123, 3214, 12)$-colorable
as shown in
$
\raisebox{1.5pt}{$4$}
\raisebox{0pt}{$8$}
\raisebox{0pt}{$3$}
\raisebox{-1.5pt}{$1$}
\raisebox{0pt}{$2$}
\raisebox{-1.5pt}{$5$}
\raisebox{1.5pt}{$6$}
\raisebox{1.5pt}{$7$}
\raisebox{0pt}{$9$}
$
where
$467 \simeq 123$,
$8329 \simeq 3214$ and
$15 \simeq 12$.
\end{example}

Given permutations $\pi$ and $\sigma^1, \sigma^2, \ldots, \sigma^k$,
the \textsc{$k$-Permutation Coloring} problem is to decide whether $\pi0$
is $(\sigma^1, \sigma^2, \ldots, \sigma^k)$ -colorable.

A \emph{separable permutation} is a permutation that has a \emph{separating tree}:
a rooted binary tree in which the elements
of the permutation appear (in permutation order) at the leaves of the tree,
and in which the descendants of each tree node form a contiguous subset of
these elements.
Each interior node of the tree is either a positive node in which all
descendants of the left child are smaller than all descendants of the right node,
or a negative node in which all descendants of the left node are greater than all
descendants of the right node.
Separable permutations may be characterized by the forbidden permutation
patterns $2413$ and $3142$.
The separable permutations are enumerated by the Schröder numbers.

\subsection*{\textbf{Linear graphs}}

A \emph{matching} $\mathcal{M}$ of size $m$, written $|\mathcal{M}| = m$, 
is a graph on the vertex set $[2m]$ whose very vertex has degree one.
For an edge of a matching, we write $(i, j)$ with $i < j$ instead of the usual $\{i, j\}$.
A matching $\mathcal{M}^0 = (V_0, E_0)$ \emph{contains} a matching 
$\mathcal{M}^1 = (V_1, E_1)$ if there exists a monotonic edge-preserving injection from $V_1$ to $V_0$.
In other words, $\mathcal{M}^0$ contains $\mathcal{M}^1$ if there exists a function 
$f : V_1 \to V_0$ such that 
(i) $1 \leq u < v \leq |\mathcal{M}|$ implies $f(u) < f(v)$ and  
(ii) $\{u, v\} \in E_1$ implies $\{f(u), f(v)\} \in E_0$.

Let $\mathcal{M}^0, \mathcal{M}^1, \dots, \mathcal{M}^k$ be $k+1$ matchings.
The matchings $\mathcal{M}^0$ is a \emph{merge} of the matchings 
$\mathcal{M}^1, \dots, \mathcal{M}^k$ 
if 
(i) $\mathcal{M}^0$ contains $\mathcal{M}^i$ for every $1 \leq i \leq k$ 
and (ii) $\left|\mathcal{M}^0\right| = \sum_{i=1}^{k} \left|\mathcal{M}^i\right|$.
In other words,  $\mathcal{M}^0 = (V_0, E_0)$ is a \emph{merge} of the matchings 
$\mathcal{M}^1, \dots, \mathcal{M}^k$ is there exist a function (\emph{i.e.} coloring)
$c : V_0 \to [k]$  and functions $f_i : V_i \to V_0$ such that, for every $1 \leq i \leq k$,
(i) $u < v$ implies $f_i(u) < f_(v)$ and  
(ii) $\{u, v\} \in E_i$ implies $\{f(u), f(v)\} \in E_0$ and $(c \circ f)(u) = (c \circ f)(v) = i$.
\textsc{$k$-Merge matching} is the problem to decide whether some matching 
$\mathcal{M}^0$ is a merge of matchings $\mathcal{M}^1, \dots, \mathcal{M}^k$.
An illustration of this definition is given in Figure~\ref{fig:matching-merge-example}.

\begin{figure}
  \centering
  \begin{tikzpicture}
    [
      scale=.75,
      vertex/.style = {circle, fill, minimum size=6pt,inner sep=0pt, outer sep=0pt},
    ]
    \draw[xstep=1,ystep=.5,black,ultra thin,black!5] (0.5,-9.25) grid (16.5,1.75);

    % matching M^0
    \draw (0,0) node {$\mathcal{M}^0$};

    % vertices
    \foreach \x in {1,...,16}
      \node [vertex,label=below:${\x}$] (vM0\x) at (\x,0) {};

    % M^1 occurrence
    \draw [edge] (vM01.north)  ..controls ($(vM01.north) + (0,1.5)$)  and ($(vM06.north) + (0,1.5)$) .. (vM06.north);
    \draw [edge] (vM02.north)  ..controls ($(vM02.north) + (0,0.5)$)  and ($(vM03.north) + (0,0.5)$)  .. (vM03.north);
    \draw [edge] (vM015.north) ..controls ($(vM015.north) + (0,0.5)$) and ($(vM016.north) + (0,0.5)$) .. (vM016.north);

    % M^2 occurrence
    \draw [edge] (vM04.north)  ..controls ($(vM04.north) + (0,1.5)$)   and ($(vM08.north) + (0,1.5)$) .. (vM08.north);
    \draw [edge] (vM010.north) ..controls ($(vM010.north) + (0,0.5)$) and ($(vM011.north) + (0,0.5)$)  .. (vM011.north);
    \draw [edge] (vM013.north) ..controls ($(vM013.north) + (0,0.5)$) and ($(vM014.north) + (0,0.5)$)  .. (vM014.north);

    % M^3 occurrence
    \draw [edge] (vM05.north)  ..controls ($(vM05.north) + (0,1.5)$)   and ($(vM09.north) + (0,1.5)$) .. (vM09.north);
    \draw [edge] (vM07.north)  ..controls ($(vM07.north) + (0,1.5)$)   and ($(vM012.north) + (0,1.5)$)  .. (vM012.north);

    % matching M^1
    \begin{scope}[yshift=-3cm]
      \draw (0,0) node {$\mathcal{M}^1$};

      \foreach \x/\l in {1/1,2/2,3/3,6/4,15/5,16/6}
        \node [vertex,fill=black,label=below:${\l}$] (vM1\l) at (\x,0) {};

      \draw [edge] (vM11.north) ..controls ($(vM11.north) + (0,1.5)$) and ($(vM14.north) + (0,1.5)$) .. (vM14.north);
      \draw [edge] (vM12.north) ..controls ($(vM12.north) + (0,0.5)$) and ($(vM13.north) + (0,0.5)$) .. (vM13.north);
      \draw [edge] (vM15.north) ..controls ($(vM15.north) + (0,0.5)$) and ($(vM16.north) + (0,0.5)$) .. (vM16.north);    
    \end{scope}

    % matching M^2
    \begin{scope}[yshift=-6cm]
      \draw (0,0) node {$\mathcal{M}^2$};

      \foreach \x/\l in {4/1,8/2,10/3,11/4,13/5,14/6}
        \node [vertex,fill=black,label=below:${\l}$] (vM2\l) at (\x,0) {};

      \draw [edge] (vM21.north) ..controls ($(vM21.north) + (0,1.5)$) and ($(vM22.north) + (0,1.5)$) .. (vM22.north);
      \draw [edge] (vM23.north) ..controls ($(vM23.north) + (0,0.5)$) and ($(vM24.north) + (0,0.5)$) .. (vM24.north);
      \draw [edge] (vM25.north) ..controls ($(vM25.north) + (0,0.5)$) and ($(vM26.north) + (0,0.5)$) .. (vM26.north);    
    \end{scope}

    % matching M^3
    \begin{scope}[yshift=-9cm]
      \draw (0,0) node {$\mathcal{M}^3$};

      \foreach \x/\l in {5/1,7/2,9/3,12/4}
        \node [vertex,fill=black,label=below:${\l}$] (vM3\l) at (\x,0) {};

      \draw [edge] (vM31.north) ..controls ($(vM31.north) + (0,1.5)$) and ($(vM33.north) + (0,1.5)$) .. (vM33.north);
      \draw [edge] (vM32.north) ..controls ($(vM32.north) + (0,1.5)$) and ($(vM34.north) + (0,1.5)$) .. (vM34.north);
    \end{scope}

  \end{tikzpicture}
  \caption{\label{fig:matching-merge-example}%
    A positive instance of \textsc{$3$-Merge matching} together with an alignment
    describing a solution.
  }% end caption
\end{figure}

