\subsubsection{Merging matchings}
\label{section:Merging matchings}

This section is devoted to proving that the
\textsc{$3$-Merge Matchings} is \NP-complete.

\begin{proposition}
  \label{proposition:3-Linear Graph Coloring is NP-complete}
  \textsc{$3$-Merge Matching} is \NP-complete.
\end{proposition}

\begin{proof}
  We reduce from the \textsc{$3$-Sat} problem, which is known to
  \NP-complete \cite{DBLP:conf/coco/Karp72}.
  Let $\phi = c_1 \wedge c_2 \wedge \cdots \wedge c_m$ be a CNF formula
  defined over the boolean variables $X =\{x_1, x_2, \dots, x_n\}$.
  For every variable $x_{i} \in X$,
  we let $\OCCURRENCE_1(x_{i})$ (resp. $\OCCURRENCE_2(x_{i})$ and $\OCCURRENCE_3(x_{i})$)
  stand for the number of occurrences of the variable $x_{i}$ as the first (resp. second and third)
  literal of a clause.
  Furthermore, we let $\OCCURRENCE(x_{i})$ stand for
  $\OCCURRENCE_1(x_{i}) + 2\,\OCCURRENCE_2(x_{i}) + \OCCURRENCE_3(x_{i})$.
  For any clause $c$, we let $c[i]$ stand for
  $i$-th literal of $c$.

  We define an instance of the \textsc{$3$-Linear Graph Coloring} problem by defining a
  target linear graph $G^{0} = (V^{0}, E^{0})$
  and three linear graphs
  $G^{1} = (V^{1}, E^{1})$, $G^{2} = (V^{2}, E^{2})$ and $G^{3} = (V^{3}, E^{3})$.
  All linear graphs $G^{0}$, $G^{1}$, $G^{2}$ and $G^{3}$ are actually composed
  of independent edges so that any connected component is composed of two vertices only.
  The general idea of the reduction is as follows:
  the target linear graph $G^{0}$ encodes the whole instance
  and $G^{1}$ encodes an assignment of the variable that satisfies all clauses of $\phi$;
  both $G^{2}$ and $G^{3}$ act as garbage collectors by focusing of those edges of $G^{0}$
  that are not part of the satisfying assignment or are structure edges.

  Set $N=m^2$.
  We have divided the presentation of the linear graphs $G^{0}, G^{1}, G^{2}$ and $G^{3}$.

  % linear graph G^0
  \medskip
  \textbf{Defining the target linear graph $G^{0} = (V^{0}, E^{0})$}
  \medskip

  % \begin{mdframed}
    Define
    \begin{alignat*} {2}
      V^{0} &= \texttt{Var}^{0} \cup \texttt{Cls}^{0}
      &\quad&\text{with $\texttt{Var}^{0} < \texttt{Cls}^{0}$,}
      \\
      \texttt{Var}^{0} &= \bigcup_{i=1}^{n} \texttt{Var}^{0}_{i}
      &&\text{with $\texttt{Var}^{0}_1 < \texttt{Var}^{0}_2 < \cdots < \texttt{Var}^{0}_n$,}
      \\
      \texttt{Cls}^{0} &= \bigcup_{j=1}^{m} \texttt{Cls}^{0}_{j}
      &&\text{with $\texttt{Cls}^{0}_1 < \texttt{Cls}^{0}_2 < \cdots < \texttt{Cls}^{0}_m$.}
    \end{alignat*}

    Let us first define the vertices of $\texttt{Var}^{0}$ that correspond
    to the variables $X$.
    Quite naturally, the vertices of $\texttt{Var}^{0}_{i}$, $1 \leq i \leq n$,
    are associated to the boolean variable $x_{i} \in X$.
    For $1 \leq i \leq n$, define
    \begin{alignat*}{2}
      \texttt{Var}^{0}_{i} &=
      &&
      \texttt{T}^{0}_{i,\LEFT} \cup \texttt{T}^{0}_{i,\RIGHT} \cup
      \texttt{F}^{0}_{i,\LEFT} \cup \texttt{F}^{0}_{i,\RIGHT} \cup
      \texttt{Y}^{0}_{i,\LEFT} \cup \texttt{Y}^{0}_{i,\RIGHT} \cup
      \texttt{X}^{0}_{i} \cup \neg\texttt{X}^{0}_{i}
      \\
      \intertext{
      with $$\texttt{T}^{0}_{i,\LEFT} <
      \texttt{Y}^{0}_{i,\LEFT} <
      \texttt{X}^{0}_{i} <
      \texttt{F}^{0}_{i,\LEFT} <
      \texttt{T}^{0}_{i,\RIGHT} <
      \neg\texttt{X}^{0}_{i} <
      \texttt{Y}^{0}_{i,\RIGHT} <
      \texttt{F}^{0}_{i,\RIGHT},$$
      and}
      \texttt{T}^{0}_{i,\LEFT}
      &=
      &&\{\texttt{t}^{0}_{i,j,\LEFT} : 1 \leq j \leq N\}
      \\
      \texttt{Y}^{0}_{i,\LEFT}
      &=
      &&\{\texttt{y}^{0}_{i,j,\LEFT} : 1 \leq j \leq N\}
      \\
      \texttt{X}^{0}_{i}
      &=
      &&\texttt{X}^{0}_{i,1} \cup \texttt{X}^{0}_{i,2} \cup \texttt{X}^{0}_{i,3}
      \\
      \texttt{X}^{0}_{i,1}
      &=
      &&\{\texttt{x}^{0}_{i,1,j} :
      \text{$c_{j}[1] = x_{i}$ or $c_{j}[1] = \overline{x_{i}}$}\}
      \\
      \texttt{X}^{0}_{i,2}
      &=
      &&\{\texttt{x}^{0}_{i,2,j,\FST}, \texttt{x}^{0}_{i,2,j,\SND} :
      \text{$c_{j}[2] = x_{i}$ or $c_{j}[2] = \overline{x_{i}}$}\}
      \\
      \texttt{X}^{0}_{i,3}
      &=
      &&\{\texttt{x}^{0}_{i,3,j} : \text{$c_{j}[3] = x_{i}$ or $c_{j}[3] = \overline{x_{i}}$}\}
      \\
      \texttt{F}^{0}_{i,\LEFT}
      &=
      &&\{\texttt{f}^{0}_{i,j,\LEFT} : 1 \leq j \leq N\}
      \\
      \texttt{T}^{0}_{i,\RIGHT}
      &=
      &&\{\texttt{t}^{0}_{i,j,\RIGHT} : 1 \leq j \leq N\}
      \\
      \neg\texttt{X}^{0}_{i}
      &=
      &&\neg\texttt{X}^{0}_{i,1} \cup \neg\texttt{X}^{0}_{i,2} \cup \neg\texttt{X}^{0}_{i,3}
      \\
      \neg\texttt{X}^{0}_{i,1}
      &=
      &&\{\neg\texttt{x}^{0}_{i,1,j} :
      \text{$c_{j}[1] = x_{i}$ or $c_{j}[1] = \overline{x_{i}}$}\}
      \\
      \neg\texttt{X}^{0}_{i,2}
      &=
      &&\{\neg\texttt{x}^{0}_{i,2,j,\FST}, \texttt{x}^{0}_{i,2,j,\SND} :
      \text{$c_{j}[2] = x_{i}$ or $c_{j}[2] = \overline{x_{i}}$}\}
      \\
      \neg\texttt{X}^{0}_{i,3}
      &=
      &&\{\neg\texttt{x}^{0}_{i,3,j} :
      \text{$c_{j}[3] = x_{i}$ or $c_{j}[3] = \overline{x_{i}}$}\}
      \\
      \texttt{Y}^{0}_{i,\RIGHT}
      &=
      &&\{\texttt{y}^{0}_{i,j,\RIGHT} : 1 \leq j \leq N\}
      \\
      \texttt{F}^{0}_{i,\RIGHT}
      &=
      &&\{\texttt{f}^{0}_{i,j,\RIGHT} : 1 \leq j \leq N\}\text{.}
    \end{alignat*}
    All vertices within subsets
    $\texttt{T}^{0}_{i,\LEFT}, \texttt{t}^{0}_{i,\RIGHT},
    \texttt{F}^{0}_{i,\LEFT}, \texttt{F}^{0}_{i,\RIGHT},
    \texttt{Y}^{0}_{i,\LEFT}, \texttt{Y}^{0}_{i,\RIGHT},
    \texttt{X}^{0}_{i}$ and $\neg\texttt{X}^{0}_{i}$
    are ordered according to their $j$-coordinate.
    Furthermore,
    $\texttt{x}^{0}_{i,2,j,\FST} < \texttt{x}^{0}_{i,2,j,\SND}$
    for $\texttt{x}^{0}_{i,2,j,\FST}, \texttt{x}^{0}_{i,2,j,\SND} \in \texttt{X}^{0}_{i,2}$
    and
    $\neg\texttt{x}^{0}_{i,2,j,\FST} < \neg\texttt{x}^{0}_{i,2,j,\SND}$
    for $\neg\texttt{x}^{0}_{i,2,j,\FST}, \neg\texttt{x}^{0}_{i,2,j,\SND} \in \neg\texttt{X}^{0}_{i,2}$.

    As for the vertices of $\texttt{Cls}^{0}$, for every $1 \leq j \leq m$, define
    \begin{align*}
      \texttt{Cls}^{0}_{j} &=
      \texttt{L}^{0}_{j} \cup
      \texttt{A}^{0}_{j} \cup
      \texttt{B}^{0}_{j} \cup
      \texttt{C}^{0}_{j} \cup
      \texttt{D}^{0}_{j}
      \\
      \texttt{L}^{0}_{j} &= \{
      \texttt{l}^{0}_{j,1,\TRUE},
      \texttt{l}^{0}_{j,1,\FALSE},
      \texttt{l}^{0}_{j,2,\FALSE,\FST},
      \texttt{l}^{0}_{j,2,\TRUE},
      \texttt{l}^{0}_{j,2,\FALSE,\SND},
      \texttt{l}^{0}_{j,3,\FALSE},
      \texttt{l}^{0}_{j,3,\TRUE}
      \}
      \\
      \texttt{A}^{0}_{j} &= \{\texttt{a}^{0}_{j,\LEFT}, \texttt{a}^{0}_{j,\RIGHT}\}
      \\
      \texttt{B}^{0}_{j} &= \{\texttt{b}^{0}_{j,\LEFT}, \texttt{b}^{0}_{j,\RIGHT}\}
      \\
      \texttt{C}^{0}_{j} &= \{\texttt{c}^{0}_{j,\LEFT}, \texttt{c}^{0}_{j,\RIGHT}\}
      \\
      \texttt{D}^{0}_{j} &= \{\texttt{d}^{0}_{j,\LEFT}, \texttt{d}^{0}_{j,\RIGHT}\}
    \end{align*}
    with
    $
    \texttt{l}^{0}_{j,1,\TRUE} <
    \texttt{a}^{0}_{j,\LEFT} <
    \texttt{l}^{0}_{j,1,\FALSE} <
    \texttt{b}^{0}_{j,\LEFT} <
    \texttt{a}^{0}_{j,\RIGHT} <
    \texttt{l}^{0}_{j,2,\FALSE,\FST} <
    \texttt{b}^{0}_{j,\RIGHT} <
    \texttt{l}^{0}_{j,2,\TRUE} <
    \texttt{c}^{0}_{j,\LEFT} <
    \texttt{l}^{0}_{j,2,\FALSE,\SND} <
    \texttt{d}^{0}_{j,\LEFT} <
    \texttt{c}^{0}_{j,\RIGHT} <
    \texttt{l}^{0}_{j,3,\FALSE} <
    \texttt{d}^{0}_{j,\RIGHT} <
    \texttt{l}^{0}_{j,3,\TRUE}
    $.

    We now turn to defining the edges $E^{0}$ of the linear graph $G^{0}$.
    Define
    \begin{alignat*}{2}
      E^{0} &=&& E^{0}_{\texttt{Var}} \cup E^{0}_{\texttt{Var},\texttt{Cls}} \cup E^{0}_{\texttt{Cls}},
      \\
      E^{0}_{\texttt{Var}} &=&& \bigcup_{i=1}^{n} E^{0}_{\texttt{Var}, i},
      \\
      \forall 1\leq i \leq n,\;
      E^{0}_{\texttt{Var}, i} &=&&
      \bigcup_{k=1}^{N} (\texttt{t}^{0}_{i,k,\LEFT}, \texttt{t}^{0}_{i,N-k+1,\RIGHT})
      \;\cup\;
      \bigcup_{k=1}^{N} (\texttt{f}^{0}_{i,k,\LEFT}, \texttt{f}^{0}_{i,N-k+1,\RIGHT}) \;\cup
      \\
      &&&
      \bigcup_{k=1}^{N} (\texttt{y}^{0}_{i,k,\LEFT}, \texttt{y}^{0}_{i,N-k+1,\RIGHT}),
      \\
      E^{0}_{\texttt{Cls}} &=&& \bigcup_{j=1}^{m} E^{0}_{\texttt{Cls}, j},
      \\
      \forall 1\leq j \leq m,\;
      E^{0}_{\texttt{Cls}, j} &=&&
      \{(\texttt{a}^{0}_{j,\LEFT}, \texttt{a}^{0}_{j,\RIGHT}),
      (\texttt{b}^{0}_{j,\LEFT}, \texttt{b}^{0}_{j,\RIGHT}),
      (\texttt{c}^{0}_{j,\LEFT}, \texttt{c}^{0}_{j,\RIGHT}),
      (\texttt{d}^{0}_{j,\LEFT}, \texttt{d}^{0}_{j,\RIGHT})\}\text{.}
    \end{alignat*}
    For $1 \leq i \leq n$, the edges of $E^{0}_{\texttt{Var}, i}$ connect
    vertices of $\texttt{Var}^{0}_{i}$ only, and
    for $1 \leq j \leq m$, the edges of $E^{0}_{\texttt{Cls}, j}$ connect
    vertices of $\texttt{Cls}^{0}_{j}$ only.
    These edges of $E^{0}_{\texttt{Var}} \cup E^{0}_{\texttt{Cls}}$ are thus called
    \emph{intra-gadget} edges (they live inside gadgets).
    What is left is thus to define the \emph{inter-gadget} edges of
    $E^{0}_{\texttt{Var},\texttt{Cls}}$.
    \begin{itemize}
      \item
      For every $\texttt{l}^{0}_{j, 1, \TRUE} \in V^{0}_{\texttt{Cls}}$,
      we add
      the edge $(\texttt{x}^{0}_{i,1,j}, \texttt{l}^{0}_{j, 1, \TRUE})$
      to $E^{0}_{\texttt{Var},\texttt{Cls}}$
      if $c_{j}[1] = x_{i}$, or
      the edge $(\neg\texttt{x}^{0}_{i,1,j}, \texttt{l}^{0}_{j, 1, \TRUE})$
      to $E^{0}_{\texttt{Var},\texttt{Cls}}$
      if $c_{j}[1] = \overline{x_{i}}$.

      \item
      For every $\texttt{l}^{0}_{j, 1, \FALSE} \in V^{0}_{\texttt{Cls}}$,
      we add
      the edge $(\neg\texttt{x}^{0}_{i,1,j}, \texttt{l}^{0}_{j, 1, \TRUE})$
      to $E^{0}_{\texttt{Var},\texttt{Cls}}$
      $c_{j}[1] = x_{i}$, or
      the edge $(\texttt{x}^{0}_{i,1,j}, \texttt{l}^{0}_{j, 1, \TRUE})$
      to $E^{0}_{\texttt{Var},\texttt{Cls}}$
      $c_{j}[1] = \overline{x_{i}}$.

      \item
      For every $\texttt{l}^{0}_{j, 2, \FALSE, \FST} \in V^{0}_{\texttt{Cls}}$, we add
      the edge $(\neg\texttt{x}^{0}_{i,2,j,\FST}, \texttt{l}^{0}_{j, 2, \FALSE, \FST})$
      to $E^{0}_{\texttt{Var},\texttt{Cls}}$
      $c_{j}[2] = x_{i}$, or
      the edge $(\texttt{x}^{0}_{i,2,j,\FST}, \texttt{l}^{0}_{j, 2, \FALSE, \FST})$
      to $E^{0}_{\texttt{Var},\texttt{Cls}}$
      $c_{j}[2] = \overline{x_{i}}$.

      \item
      For every $\texttt{l}^{0}_{j, 2, \FALSE, \SND} \in V^{0}_{\texttt{Cls}}$, we add
      the edge $(\neg\texttt{x}^{0}_{i,2,j,\SND}, \texttt{l}^{0}_{j, 2, \FALSE, \SND})$
      to $E^{0}_{\texttt{Var},\texttt{Cls}}$
      $c_{j}[2] = x_{i}$, or
      the edge $(\texttt{x}^{0}_{i,2,j,\SND}, \texttt{l}^{0}_{j, 2, \FALSE, \SND})$
      to $E^{0}_{\texttt{Var},\texttt{Cls}}$
      $c_{j}[2] = \overline{x_{i}}$.

      \item
      For every $\texttt{l}^{0}_{j, 3, \TRUE} \in V^{0}_{\texttt{Cls}}$,
      we add
      the edge $(\texttt{x}^{0}_{i,3,j}, \texttt{l}^{0}_{j, 3, \TRUE})$
      to $E^{0}_{\texttt{Var},\texttt{Cls}}$
      $c_{j}[3] = x_{i}$, or
      the edge $(\neg\texttt{x}^{0}_{i,1,j}, \texttt{l}^{0}_{j, 3, \TRUE})$
      to $E^{0}_{\texttt{Var},\texttt{Cls}}$
      $c_{j}[3] = \overline{x_{i}}$.

      \item
      For every $\texttt{l}^{0}_{j, 3, \FALSE} \in V^{0}_{\texttt{Cls}}$,
      we add
      the edge $(\neg\texttt{x}^{0}_{i,3,j}, \texttt{l}^{0}_{j, 3, \TRUE})$
      to $E^{0}_{\texttt{Var},\texttt{Cls}}$
      $c_{j}[3] = x_{i}$, or
      the edge $(\texttt{x}^{0}_{i,3,j}, \texttt{l}^{0}_{j, 3, \TRUE})$
      to $E^{0}_{\texttt{Var},\texttt{Cls}}$
      $c_{j}[3] = \overline{x_{i}}$.
    \end{itemize}
  % \end{mdframed}
  
  \medskip

  The construction of the linear Graph $G^{0}$ is illustrated
  in Figure~\ref{fig-3-linear-graph-splitting-variable-gadget-0} and
  Figure~\ref{fig-3-linear-graph-splitting-clause-gadget-0}.

  \begin{figure}
  \centering

  \begin{tikzpicture}
    [
    scale=.375,
    my vertex/.style={basic vertex,
    minimum size=1pt,
    inner sep=2pt,
    ball color=black!80},
    ]
    % vertices

    % p left
    \begin{scope}[]
      \fill [rounded corners, black!15] (-0.6,-0.6) -- (-0.6,0.6) --
      (3.6,0.6)   -- (3.6,-0.6) --
      cycle;
      \node [
      my vertex,
      label={[text depth=0ex,label distance=0.25cm,rotate=-90]right:{\scriptsize $\texttt{t}^{0}_{i,1,\LEFT}$}}
      ] (P1l) at (0,0) {};
      \node [
      my vertex,
      label={[text depth=0ex,label distance=0.25cm,rotate=-90]right:{\scriptsize $\texttt{t}^{0}_{i,2,\LEFT}$}}
      ] (P2l) at (1,0) {};
      \node [
      my vertex,
      label={[text depth=0ex,label distance=0.25cm,rotate=-90]right:{\scriptsize $\texttt{t}^{0}_{i,N,\LEFT}$}}
      ] (PNl) at (3,0) {};
      \path [draw,dotted] (P2l) -- (PNl);
    \end{scope}

    % r left
    \begin{scope}[xshift=4.5cm]
      \fill [rounded corners, black!15] (-0.6,-0.6) -- (-0.6,0.6) --
      (3.6,0.6)   -- (3.6,-0.6) --
      cycle;
      \node [
      my vertex,
      label={[text depth=0ex,label distance=0.25cm,rotate=-90]right:{\scriptsize $\texttt{y}^{0}_{i,1,\LEFT}$}}
      ] (R1l) at (0,0) {};
      \node [
      my vertex,
      label={[text depth=0ex,label distance=0.25cm,rotate=-90]right:{\scriptsize $\texttt{y}^{0}_{i,2,\LEFT}$}}
      ] (R2l) at (1,0) {};
      \node [
      my vertex,
      label={[text depth=0ex,label distance=0.25cm,rotate=-90]right:{\scriptsize $\texttt{y}^{0}_{i,N,\LEFT}$}}
      ] (RNl) at (3,0) {};
      \path [draw,dotted] (R2l) -- (RNl);
    \end{scope}

    % x
    \begin{scope}[xshift=9cm]
      \fill [rounded corners, black!15] (-0.6,-0.6) -- (-0.6,0.6) --
      (3.6,0.6)   -- (3.6,-0.6) --
      cycle;
      \node [
      my vertex,
      label={[text depth=2.5ex,label distance=0.25cm,rotate=-90]left:{\scriptsize $\texttt{x}^{0}_{i,1}$}}
      ] (X1) at (0,0) {};
      \node [
      my vertex,
      label={[text depth=2.5ex,label distance=0.25cm,rotate=-90]left:{\scriptsize $\texttt{x}^{0}_{i,2}$}}
      ] (X2) at (1,0) {};
      \node [
      my vertex,
      label={[text depth=2.5ex,label distance=0.25cm,rotate=-90]left:{\scriptsize $\texttt{x}^{0}_{i,k}$}}
      ] (Xk) at (3,0) {};
      \path [draw,dotted] (X2) -- (Xk);
    \end{scope}

    % q left
    \begin{scope}[xshift=13.5cm]
      \fill [rounded corners, black!15] (-0.6,-0.6) -- (-0.6,0.6) --
      (3.6,0.6)   -- (3.6,-0.6) --
      cycle;
      \node [
      my vertex,
      label={[text depth=0ex,label distance=0.25cm,rotate=-90]right:{\scriptsize $\texttt{f}^{0}_{i,1,\LEFT}$}}
      ] (Q1l) at (0,0) {};
      \node [
      my vertex,
      label={[text depth=0ex,label distance=0.25cm,rotate=-90]right:{\scriptsize $\texttt{f}^{0}_{i,2,\LEFT}$}}
      ] (Q2l) at (1,0) {};
      \node [
      my vertex,
      label={[text depth=0ex,label distance=0.25cm,rotate=-90]right:{\scriptsize $\texttt{f}^{0}_{i,N,\LEFT}$}}
      ] (QNl) at (3,0) {};
      \path [draw,dotted] (Q2l) -- (QNl);
    \end{scope}

    % p right
    \begin{scope}[xshift=18cm]
      \fill [rounded corners, black!15] (-0.6,-0.6) -- (-0.6,0.6) --
      (3.6,0.6)   -- (3.6,-0.6) --
      cycle;
      \node [
      my vertex,
      label={[text depth=0ex,label distance=0.25cm,rotate=-90]right:{\scriptsize $\texttt{t}^{0}_{i,1,\RIGHT}$}}
      ] (P1r) at (0,0) {};
      \node [
      my vertex,
      label={[text depth=0ex,label distance=0.25cm,rotate=-90]right:{\scriptsize $\texttt{t}^{0}_{i,2,\RIGHT}$}}
      ] (P2r) at (1,0) {};
      \node [
      my vertex,
      label={[text depth=0ex,label distance=0.25cm,rotate=-90]right:{\scriptsize $\texttt{t}^{0}_{i,N,\RIGHT}$}}
      ] (PNr) at (3,0) {};
      \path [draw,dotted] (P2r) -- (PNr);
    \end{scope}

    % neg x
    \begin{scope}[xshift=22.5cm]
      \fill [rounded corners, black!15] (-0.6,-0.6) -- (-0.6,0.6) --
      (3.6,0.6)   -- (3.6,-0.6) --
      cycle;
      \node [
      my vertex,
      label={[text depth=2.5ex,label distance=0.25cm,rotate=-90]left:{\scriptsize $\neg\texttt{x}^{0}_{i,1}$}}
      ] (nX1) at (0,0) {};
      \node [
      my vertex,
      label={[text depth=2.5ex,label distance=0.25cm,rotate=-90]left:{\scriptsize $\neg\texttt{x}^{0}_{i,2}$}}
      ] (nX2) at (1,0) {};
      \node [
      my vertex,
      label={[text depth=2.5ex,label distance=0.25cm,rotate=-90]left:{\scriptsize $\neg\texttt{x}^{0}_{i,k}$}}
      ] (nXk) at (3,0) {};
      \path [draw,dotted] (nX2) -- (nXk);
    \end{scope}

    % r right
    \begin{scope}[xshift=27cm]
      \fill [rounded corners, black!15] (-0.6,-0.6) -- (-0.6,0.6) --
      (3.6,0.6)   -- (3.6,-0.6) --
      cycle;
      \node [
      my vertex,
      label={[text depth=0ex,label distance=0.25cm,rotate=-90]right:{\scriptsize $\texttt{y}^{0}_{i,1,\RIGHT}$}}
      ] (R1r) at (0,0) {};
      \node [
      my vertex,
      label={[text depth=0ex,label distance=0.25cm,rotate=-90]right:{\scriptsize $\texttt{y}^{0}_{i,2,\RIGHT}$}}
      ] (R2r) at (1,0) {};
      \node [
      my vertex,
      label={[text depth=0ex,label distance=0.25cm,rotate=-90]right:{\scriptsize $\texttt{y}^{0}_{i,N,\RIGHT}$}}
      ] (RNr) at (3,0) {};
      \path [draw,dotted] (R2r) -- (RNr);
    \end{scope}

    % q right
    \begin{scope}[xshift=31.5cm]
      \fill [rounded corners, black!15] (-0.6,-0.6) -- (-0.6,0.6) --
      (3.6,0.6)   -- (3.6,-0.6) --
      cycle;
      \node [
      my vertex,
      label={[text depth=0ex,label distance=0.25cm,rotate=-90]right:{\scriptsize $\texttt{f}^{0}_{i,1,\RIGHT}$}}
      ] (Q1r) at (0,0) {};
      \node [
      my vertex,
      label={[text depth=0ex,label distance=0.25cm,rotate=-90]right:{\scriptsize $\texttt{f}^{0}_{i,2,\RIGHT}$}}
      ] (Q2r) at (1,0) {};
      \node [
      my vertex,
      label={[text depth=0ex,label distance=0.25cm,rotate=-90]right:{\scriptsize $\texttt{f}^{0}_{i,N,\RIGHT}$}}
      ] (QNr) at (3,0) {};
      \path [draw,dotted] (Q2r) -- (QNr);
    \end{scope}

    % edges
    \path [edge]
    (P1l.north) -- ++(0,5.5) -- ($(PNr.north)+(0,5.5)$) -- (PNr.north);
    \path [edge,path fading=east]
    (P2l.north) -- ++(0,5) --  ($(P2r.north)+(0,5)$);
    \path [edge,path fading=west]
    (P2r.north) -- ++(0,4) -- ($(PNl.north)+(0,4)$);
    \path [edge]
    (PNl.north) -- ($(PNl.north)+(0,3.5)$) -- ($(P1r.north)+(0,3.5)$) -- (P1r.north);

    \path [edge]
    (Q1l.north) -- ++(0,13) -- ($(QNr.north)+(0,13)$) -- (QNr.north);
    \path [edge,path fading=east]
    (Q2l.north) -- ++(0,12.5) --  ($(Q2r.north)+(0,12.5)$);
    \path [edge,path fading=west]
    (Q2r.north) -- ++(0,11.5) -- ($(QNl.north)+(0,11.5)$);
    \path [edge]
    (QNl.north) -- ($(QNl.north)+(0,10.5)$) -- ($(Q1r.north)+(0,10.5)$) -- (Q1r.north);

    \path [edge]
    (R1l.north) -- ++(0,9) -- ($(RNr.north)+(0,9)$) -- (RNr.north);
    \path [edge,path fading=east]
    (R2l.north) -- ++(0,8.5) --  ($(R2r.north)+(0,8.5)$);
    \path [edge,path fading=west]
    (R2r.north) -- ++(0,7.5) -- ($(RNl.north)+(0,7.5)$);
    \path [edge]
    (RNl.north) -- ($(RNl.north)+(0,7)$) -- ($(R1r.north)+(0,7)$) -- (R1r.north);

    \path [edge,path fading=east] (X1.south) -- ++(0,-5) -- ++(18,0);
    \path [edge,path fading=east] (X2.south) -- ++(0,-5.5) -- ++(18,0);
    \path [edge,path fading=east] (Xk.south) -- ++(0,-6.5) -- ++(17,0);

    \path [edge,path fading=east] (nX1.south) -- ++(0,-7) -- ++(7.5,0);
    \path [edge,path fading=east] (nX2.south) -- ++(0,-7.5) -- ++(7.5,0);
    \path [edge,path fading=east] (nXk.south) -- ++(0,-8.5) -- ++(6.5,0);

    %\node [text width=2cm,anchor=west] at (31,-7) {\scriptsize Towards the clause gadgets};
  \end{tikzpicture}

  \caption{\label{fig-3-linear-graph-splitting-variable-gadget-0}%
  A variable gadget $\texttt{Var}^{0}_{i}$ where
  $\texttt{x}^{0}_{i,1}, \texttt{x}^{0}_{i,2}, \dots, \texttt{x}^{0}_{i,k}$
  (resp.
  $\neg\texttt{x}^{0}_{i,1}, \neg\texttt{x}^{0}_{i,2}, \dots, \neg\texttt{x}^{0}_{i,k}$
  )
  represent the vertices of $\texttt{X}^{0}_{i}$ (resp. $\neg\texttt{X}^{0}_{i}$) with
  $k = \OCCURRENCE(x_i) = \OCCURRENCE_1(x_i) + 2\,\OCCURRENCE_2(x_i) + \OCCURRENCE_3(x_i)$.
  The intra-gadget edges of $E^{0}_{\texttt{Var}, i} =
  E^{0}_{\texttt{Var}, i, \texttt{t}} \cup
  E^{0}_{\texttt{Var}, i, \texttt{f}} \cup
  E^{0}_{\texttt{Var}, i, \texttt{y}}$ are shown together
  with the left endpoint parts of the associated inter-gadget edges of
  $E^{0}_{\texttt{Var},\texttt{Cls}}$.
  }% end caption
\end{figure}


  \begin{figure}
  \centering

  \begin{tikzpicture}
    [
      scale=.2,
      my vertex/.style={basic vertex,
      minimum size=1pt,
      inner sep=2pt,
      ball color=black!80},
    ]
    \path [draw,color=black!5] (0,0) -- (56,0);

    % vertices
    \foreach \i/\l [evaluate=\i as \x using \i*3.5] in
    {
      2/\texttt{a}^{0}_{j,\LEFT},
      4/\texttt{b}^{0}_{j,\LEFT},
      5/\texttt{a}^{0}_{j,\RIGHT},
      7/\texttt{b}^{0}_{j,\RIGHT},
      9/\texttt{c}^{0}_{j,\LEFT},
      11/\texttt{d}^{0}_{j,\LEFT},
      12/\texttt{c}^{0}_{j,\RIGHT},
      14/\texttt{d}^{0}_{j,\RIGHT}
    }
    {
    \node [
    my vertex,
    label={[text depth=0ex,anchor=center,label distance=0.15cm,rotate=-90]right:{\scriptsize $\l$}}
    ] (V\i) at (\x,0) {};
    }
    \foreach \i/\l [evaluate=\i as \x using \i*3.5] in
    {
      1/\texttt{l}^{0}_{j,1,\TRUE},
      3/\texttt{l}^{0}_{j,1,\FALSE},
      6/\texttt{l}^{0}_{j,2,\FALSE,\FST},
      8/\texttt{l}^{0}_{j,2,\TRUE},
      10/\texttt{l}^{0}_{j,2,\FALSE,\SND},
      13/\texttt{l}^{0}_{j,3,\FALSE},
      15/\texttt{l}^{0}_{j,3,\TRUE}
    }
    {

    \node [
      my vertex,
      label={[text depth=2.5ex,anchor=center,label distance=0.15cm,rotate=-90]left:{\scriptsize $\l$}}
    ] (V\i) at (\x,0) {};
    }

    % edges
    \path [edge] (V2.north) -- ++ (0,6) -- ($(V5.north)+(0,6)$) -- (V5.north);
    \path [edge] (V4.north) -- ++ (0,7) -- ($(V7.north)+(0,7)$) -- (V7.north);
    \path [edge] (V9.north) -- ++ (0,7) -- ($(V12.north)+(0,7)$) -- (V12.north);
    \path [edge] (V11.north) -- ++ (0,6) -- ($(V14.north)+(0,6)$) -- (V14.north);

    % incoming edges
    \draw [edge,path fading=west] (V1) -- ++(0,-5) -- +(-3,0) node {};
    \draw [edge,path fading=west] (V3) -- ++(0,-6) -- +(-10.5,0) node {};
    \draw [edge,path fading=west] (V6)  -- ++(0,-7)  -- +(-21.5,0) node {};
    \draw [edge,path fading=west] (V8)  -- ++(0,-8)  -- +(-29,0) node {};
    \draw [edge,path fading=west] (V10) -- ++(0,-9) -- +(-36.5,0) node {};
    \draw [edge,path fading=west] (V13) -- ++(0,-10) -- +(-47.5,0) node {};
    \draw [edge,path fading=west] (V15) -- ++(0,-11) -- +(-55,0) node {};

    \node [text width=2.5cm,anchor=west] at (-15.5,-8.5) {\scriptsize Towards the variable gadgets};

  \end{tikzpicture}

  \caption{\label{fig-3-linear-graph-splitting-clause-gadget-0}%
  A clause gadget $\texttt{Cls}^{0}_{j}$.
  The intra-gadget edges of
  $E^{0}_{\texttt{Cls}, j} =
  E^{0}_{\texttt{Cls}, j, \texttt{a}} \cup
  E^{0}_{\texttt{Cls}, j, \texttt{b}} \cup
  E^{0}_{\texttt{Cls}, j, \texttt{c}} \cup
  E^{0}_{\texttt{Cls}, j, \texttt{d}}$ are shown together
  with the right endpoint parts of the associated inter-gadget edges of
  $E^{0}_{\texttt{Var},\texttt{Cls}}$.
  }% end caption
\end{figure}


  % linear graph G^1
  \medskip
  \textbf{Defining the target linear graph $G^{1} = (V^{1}, E^{1})$}
  \medskip
  % \begin{mdframed}
    Define
    \begin{alignat*} {2}
      V^{1} &= \texttt{Var}^{1} \cup \texttt{Cls}^{1}
      &\quad&\text{with $\texttt{Var}^{1} < \texttt{C}^{1}$}
      \\
      \texttt{Var}^{1} &= \bigcup_{i=1}^{n} \texttt{Var}^{1}_{i}
      &&\text{with $\texttt{Var}^{1}_1 < \texttt{Var}^{1}_2 < \cdots < \texttt{Var}^{1}_n$}
      \\
      \texttt{Cls}^{1} &= \bigcup_{j=1}^{m} \texttt{Cls}^{1}_{j}
      &&\text{with $\texttt{Cls}^{1}_1 < \texttt{Cls}^{1}_2 < \cdots < \texttt{Cls}^{1}_m$.}
    \end{alignat*}
    For $1 \leq i \leq n$, define
    \begin{alignat*}{2}
      \texttt{Var}^{1}_{i} &=
      \texttt{TF}^{1}_{i,\LEFT} \cup \texttt{X}^{1}_{i} \cup \texttt{TF}^{1}_{i,\RIGHT}
      &&\text{with $\texttt{TF}^{1}_{i,\LEFT} < \texttt{X}^{1}_{i} < \texttt{TF}^{1}_{i,\RIGHT}$},
      \\
      \texttt{TF}^{1}_{i,\LEFT}
      &=
      \{\texttt{tf}^{1}_{i,j,\LEFT} : 1 \leq j \leq N\},
      &&
      \\
      \texttt{X}^{1}_{i}
      &=
      \{\texttt{x}^{1}_{i,j} : \text{$x_{i}$ or $\overline{x_{i}}$ occurs in clause $c_{j}$}\},
      &&
      \\
      \texttt{TF}^{1}_{i,\RIGHT}
      &=
      \{\texttt{tf}^{1}_{i,j,\RIGHT} : 1 \leq j \leq N\},
      &&
      \\
      \texttt{Cls}^{0}_{j}
      &=
      \{
      \texttt{l}^{1}_{j,1},
      \texttt{l}^{1}_{j,2},
      \texttt{l}^{1}_{j,3},
      \texttt{ab}^{1}_{j,\LEFT},
      \texttt{cd}^{1}_{j,\LEFT},
      \texttt{ab}^{1}_{j,\RIGHT},
      \texttt{cd}^{1}_{j,\RIGHT}
      \}
      &&
    \end{alignat*}
    with
    $
    \texttt{l}^{1}_{j,1} <
    \texttt{ab}^{1}_{j,\LEFT} <
    \texttt{ab}^{1}_{j,\RIGHT} <
    \texttt{l}^{1}_{j,2} <
    \texttt{cd}^{1}_{j,\LEFT} <
    \texttt{cd}^{1}_{j,\RIGHT} <
    \texttt{l}^{1}_{j,3}
    $.
    The vertices in $\texttt{Var}^{1}_{i}$ are ordered according to their second coordinate
    (always written $j$ in the above definitions).

    We now turn to defining the edges $E^{1}$ of the linear graph $G^{1}$.
    Define
    \begin{align*}
      E^{1} &= E^{1}_{\texttt{Var}} \cup E^{1}_{\texttt{Var},\texttt{Cls}} \cup E^{1}_{\texttt{Cls}},
      \\
      E^{1}_{\texttt{Var}} &= \bigcup_{i=1}^{n} E^{1}_{\texttt{Var}, i},
      \\
      \forall 1\leq i \leq n,\quad
      E^{1}_{\texttt{Var}, i} &= \bigcup_{k=1}^{N} (\texttt{TF}^{1}_{i,k}, \texttt{TF}^{1}_{i,N-k+1}),
      \\
      E^{1}_{\texttt{Cls}} &= \bigcup_{j=1}^{m} E^{1}_{\texttt{Cls}, j},
      \\
      \forall 1\leq j \leq m,\quad
      E^{1}_{\texttt{Cls}, j} &=
      \{
      (\texttt{ab}^{1}_{j,\LEFT}, \texttt{ab}^{1}_{j,\RIGHT}),
      (\texttt{cd}^{1}_{j,\LEFT}, \texttt{cd}^{1}_{j,\RIGHT})
      \}\text{.}
    \end{align*}
    What is left is to define $E^{1}_{\texttt{Var},\texttt{Cls}}$.
    For every $\texttt{l}^{1}_{j, 1} \in V^{1}_{\texttt{Cls}}$
    (resp. $\texttt{l}^{1}_{j, 2} \in V^{1}_{\texttt{Cls}}$ and
    $\texttt{l}^{1}_{j, 3} \in V^{1}_{\texttt{Cls}}$)
    we add
    the edge $(\texttt{x}^{1}_{i,j}, \texttt{l}^{1}_{j, 1})$
    (resp. $(\texttt{x}^{2}_{i,j}, \texttt{l}^{1}_{j, 1})$ and
    $(\texttt{x}^{3}_{i,j}, \texttt{l}^{1}_{j, 1})$)
    to $E^{1}_{\texttt{Var},\texttt{Cls}}$
    if the first (resp. second and third) literal of the clause $c_{j}$
    is the positive or the negative literal $x_{i}$.
  % \end{mdframed}

  The construction of the linear Graph $G^{0}$ is illustrated
  in Figure~\ref{fig-3-linear-graph-splitting-variable-gadget-1} and
  Figure~\ref{fig-3-linear-graph-splitting-clause-gadget-1}.

  \input{figures/3-linear-graph-splitting-variable-gadget-1}

  \begin{figure}
  \centering

  \begin{tikzpicture}
  [
    scale=.2,
    my vertex/.style={basic vertex,
                      minimum size=1pt,
                      inner sep=2pt,
                      ball color=black!80},
  ]
    % vertices
    \foreach \i/\l [evaluate=\i as \x using \i*3.5] in
      {
        2/\texttt{ab}^{1}_{j,\LEFT},
        3/\texttt{ab}^{1}_{j,\RIGHT},
        5/\texttt{cd}^{1}_{j,\LEFT},
        6/\texttt{cd}^{1}_{j,\RIGHT}
      }
    {
    \node [
      my vertex,
      label={[text depth=0ex,anchor=center,label distance=0.15cm,rotate=-90]right:{\scriptsize $\l$}}
    ] (V\i) at (\x,0) {};
    }

    % vertices
    \foreach \i/\l [evaluate=\i as \x using \i*3.5] in
      {
        1/\texttt{l}^{1}_{j,1},
        4/\texttt{l}^{1}_{j,2},
        7/\texttt{l}^{1}_{j,3}
      }
    {
    \node [
      my vertex,
      label={[text depth=2.5ex,anchor=center,label distance=0.15cm,rotate=-90]left:{\scriptsize $\l$}}
    ] (V\i) at (\x,0) {};    }

    % edges
    \path [edge] (V2.north) -- ++ (0,4) -- ($(V3.north)+(0,4)$) -- (V3.north);
    \path [edge] (V5.north) -- ++ (0,4) -- ($(V6.north)+(0,4)$) -- (V6.north);

    % incoming edges
    \draw [edge,path fading=west] (V1) -- ++(0,-5) -- +(-3,0) node {};
    \draw [edge,path fading=west] (V4) -- ++(0,-6) -- +(-14,0) node {};
    \draw [edge,path fading=west] (V7)  -- ++(0,-7)  -- +(-25,0) node {};

    \node [text width=2.5cm,anchor=west] at (-15.5,-8.5) {\scriptsize Towards the variable gadgets};

  \end{tikzpicture}

  \caption{\label{fig-3-linear-graph-splitting-clause-gadget-1}%
    A clause gadget $\texttt{Cls}^{1}_{j}$.
    The two intra-gadget edges of
    $E^{1}_{\texttt{Cls}, j} =
    \{(\texttt{ab}^{1}_{j, \LEFT}, \texttt{ab}^{1}_{j, \RIGHT}),
    (\texttt{cd}^{1}_{j, \LEFT}, \texttt{cd}^{1}_{j, \RIGHT})\}$
    are shown together
    with the right endpoint parts of the associated inter-gadget edges of
    $E^{1}_{\texttt{Var},\texttt{Cls}}$.
  }% end caption
\end{figure}


  % linear graph G^2
  \medskip
  \textbf{Defining the target linear graph $G^{2} = (V^{2}, E^{2})$}
  \medskip

  % \begin{mdframed}
    The linear graph $G^{2}$ is identical to $G^{1}$ except that it does not
    contain any edge in $E^{2}_{\texttt{Cls}}$.
    More formally,
    define
    \begin{alignat*} {2}
      V^{2} &= \texttt{Var}^{2} \cup \texttt{Cls}^{2}
      &\quad&\text{with $\texttt{Var}^{2} < \texttt{Cls}^{2}$}
      \\
      \texttt{Var}^{2} &= \bigcup_{i=1}^{n} \texttt{Var}^{2}_{i}
      &&\text{with $\texttt{Var}^{2}_1 < \texttt{Var}^{2}_2 < \dots < \texttt{Var}^{2}_n$}
      \\
      \texttt{Cls}^{2} &= \bigcup_{j=1}^{m} \texttt{Cls}^{2}_{j}
      &&\text{with $\texttt{Cls}^{2}_1 < \texttt{Cls}^{2}_2 < \dots < \texttt{Cls}^{2}_m$.}
      \\
      \forall, 1 \leq i \leq n,\;
      \texttt{Var}^{2}_{i} &= \texttt{TF}^{2}_{i,\LEFT} \cup \texttt{X}^{2}_{i} \cup \texttt{TF}^{2}_{i,\RIGHT}
      &&\text{with $\texttt{TF}^{2}_{i,\LEFT} < \texttt{X}^{2}_{i} < \texttt{TF}^{2}_{i,\RIGHT}$}
      \\
      \texttt{TF}^{2}_{i,\LEFT}
      &=
      \{\texttt{TF}^{2}_{i,j,\LEFT} : 1 \leq j \leq N\}
      &&\\
      \texttt{X}^{2}_{i}
      &=
      \{\texttt{x}^{2}_{i,j} : \text{$x_{i}$ or $\overline{x_{i}}$ occurs in $c_{j}$}\}
      &&\\
      \texttt{TF}^{2}_{i,\RIGHT}
      &=
      \{\texttt{TF}^{2}_{i,j,\RIGHT} : 1 \leq j \leq N\}
      &&
      \\
      \texttt{Cls}^{2}_{j}
      &=
      \{
      \texttt{l}^{2}_{j,1},
      \texttt{l}^{2}_{j,2},
      \texttt{l}^{2}_{j,3}
      \}
      &&\text{with $\texttt{l}^{2}_{j,1} < \texttt{l}^{2}_{j,2} < \texttt{l}^{2}_{j,3}$}
      \\
    \end{alignat*}
    The vertices in $\texttt{Var}^{2}_{i}$ are ordered according to their second coordinate
    (always written $j$ in the above definitions).

    We now turn to defining the edges $E^{2}$ of the linear graph $G^{2}$.
    Define
    \begin{align*}
      E^{2} &= E^{2}_{\texttt{Var}} \cup E^{2}_{\texttt{Var},\texttt{Cls}},
      \\
      E^{2}_{\texttt{Var}} &= \bigcup_{i=1}^{n} E^{2}_{\texttt{Var}, i},
      \\
      \forall 1\leq i \leq n,\quad
      E^{2}_{\texttt{Var}, i} &= \bigcup_{k=1}^{N} (\texttt{TF}^{2}_{i,k}, \texttt{TF}^{2}_{i,N-k+1})\text{.}
    \end{align*}
    As for $E^{2}_{\texttt{Var},\texttt{Cls}}$,
    \begin{itemize}
      \item
      for every $\texttt{l}^{2}_{j, 1} \in V^{2}_{\texttt{Cls}}$,
      we add
      the edge $(\texttt{x}^{2}_{i,j}, \texttt{l}^{2}_{j, 1})$
      to $E^{2}_{\texttt{Var},\texttt{Cls}}$
      if $c_{j}[1] = x_{i}$ or $c_{j}[1] = \overline{x_{i}}$.
      \item
      for every $\texttt{l}^{2}_{j, 2} \in V^{2}_{\texttt{Cls}}$,
      we add
      the edge $(\texttt{x}^{2}_{i,j}, \texttt{l}^{2}_{j, 1})$
      to $E^{2}_{\texttt{Var},\texttt{Cls}}$
      if $c_{j}[2] = x_{i}$ or $c_{j}[2] = \overline{x_{i}}$.
      \item
      for every $\texttt{l}^{2}_{j, 3} \in V^{2}_{\texttt{Cls}}$,
      we add
      the edge $(\texttt{x}^{3}_{i,j}, \texttt{l}^{2}_{j, 1})$
      to $E^{2}_{\texttt{Var},\texttt{Cls}}$
      if $c_{j}[3] = x_{i}$ or $c_{j}[3] = \overline{x_{i}}$.
    \end{itemize}
  % \end{mdframed}

  The construction of the linear graph $G^{2}$ is illustrated
  in Figure~\ref{fig-3-linear-graph-splitting-variable-gadget-2}.

  \begin{figure}
  \centering

  \begin{tikzpicture}
    [
    scale=.375,
    my vertex/.style={basic vertex,
    minimum size=1pt,
    inner sep=2pt,
    ball color=black!80},
    ]
    % vertices

    % TF left
    \begin{scope}[]
      \fill [rounded corners, black!15] (-0.6,-0.6) -- (-0.6,0.6) --
      (3.6,0.6)   -- (3.6,-0.6) --
      cycle;
      \node [
      my vertex,
      label={[text depth=0ex,label distance=0.25cm,rotate=-90]right:{\scriptsize $\texttt{tf}^{2}_{i,1,\LEFT}$}}
      ] (TF1l) at (0,0) {};
      \node [
      my vertex,
      label={[text depth=0ex,label distance=0.25cm,rotate=-90]right:{\scriptsize $\texttt{tf}^{2}_{i,2,\LEFT}$}}
      ] (TF2l) at (1,0) {};
      \node [
      my vertex,
      label={[text depth=0ex,label distance=0.25cm,rotate=-90]right:{\scriptsize $\texttt{tf}^{2}_{i,N,\LEFT}$}}
      ] (TFNl) at (3,0) {};
      \path [draw,dotted] (TF2l) -- (TFNl);
    \end{scope}

    % x
    \begin{scope}[xshift=4.5cm]
      \fill [rounded corners, black!15] (-0.6,-0.6) -- (-0.6,0.6) --
      (3.6,0.6)   -- (3.6,-0.6) --
      cycle;
      \node [
      my vertex,
      label={[text depth=2.5ex,label distance=0.25cm,rotate=-90]left:{\scriptsize $\texttt{x}^{2}_{i,1}$}}
      ] (X1) at (0,0) {};
      \node [
      my vertex,
      label={[text depth=2.5ex,label distance=0.25cm,rotate=-90]left:{\scriptsize $\texttt{x}^{2}_{i,2}$}}
      ] (X2) at (1,0) {};
      \node [
      my vertex,
      label={[text depth=2.5ex,label distance=0.25cm,rotate=-90]left:{\scriptsize $\texttt{x}^{2}_{i,k}$}}
      ] (Xk) at (3,0) {};
      \path [draw,dotted] (X2) -- (Xk);
    \end{scope}

    % p right
    \begin{scope}[xshift=9cm]
      \fill [rounded corners, black!15] (-0.6,-0.6) -- (-0.6,0.6) --
      (3.6,0.6)   -- (3.6,-0.6) --
      cycle;
      \node [
      my vertex,
      label={[text depth=0ex,label distance=0.25cm,rotate=-90]right:{\scriptsize $\texttt{tf}^{2}_{i,1,\RIGHT}$}}
      ] (TF1r) at (0,0) {};
      \node [
      my vertex,
      label={[text depth=0ex,label distance=0.25cm,rotate=-90]right:{\scriptsize $\texttt{tf}^{2}_{i,2,\RIGHT}$}}
      ] (TF2r) at (1,0) {};
      \node [
      my vertex,
      label={[text depth=0ex,label distance=0.25cm,rotate=-90]right:{\scriptsize $\texttt{tf}^{2}_{i,N,\RIGHT}$}}
      ] (TFNr) at (3,0) {};
      \path [draw,dotted] (TF2r) -- (TFNr);
    \end{scope}

    % edges
    \path [edge]
    (TF1l.north) -- ++(0,5.5) -- ($(TFNr.north)+(0,5.5)$) -- (TFNr.north);
    \path [edge,path fading=east]
    (TF2l.north) -- ++(0,5) --  ($(TF2r.north)+(0,5)$);
    \path [edge,path fading=west]
    (TF2r.north) -- ++(0,4) -- ($(TFNl.north)+(0,4)$);
    \path [edge]
    (TFNl.north) -- ($(PNl.north)+(0,3.5)$) -- ($(TF1r.north)+(0,3.5)$) -- (TF1r.north);

    \path [edge,path fading=east] (X1.south) -- ++(0,-5) -- ++(13,0);
    \path [edge,path fading=east] (X2.south) -- ++(0,-5.5) -- ++(13,0);
    \path [edge,path fading=east] (Xk.south) -- ++(0,-6.5) -- ++(12,0);

    %\node [text width=2cm,anchor=west] at (19,-6) {\scriptsize Towards the clause gadgets};
  \end{tikzpicture}

  \caption{\label{fig-3-linear-graph-splitting-variable-gadget-2}%
  A variable gadget $\texttt{Var}^{2}_{i}$ where
  $\texttt{x}^{2}_{i,1}, \texttt{x}^{2}_{i,2}, \dots, \texttt{x}^{2}_{i,k}$
  represent the vertices of $\texttt{X}^{2}_{i}$ with
  $k = \OCCURRENCE(x_i) = \OCCURRENCE_1(x_i) + 2\,\OCCURRENCE_2(x_i) + \OCCURRENCE_3(x_i)$.
  The intra-gadget edges of $E^{2}_{\texttt{Var}, i} =
  E^{2}_{\texttt{Var}, i, \texttt{tf}} $ are shown together
  with the left endpoint parts of the associated inter-gadget edges of
  $E^{2}_{\texttt{Var},\texttt{Cls}}$.
  }% end caption
\end{figure}


  % linear graph G^3
  \medskip
  \textbf{Defining the target linear graph $G^{3} = (V^{3}, E^{3})$}
  \medskip

  % \begin{mdframed}
    Define
    \begin{alignat*}{2}
      V^{3} &= \texttt{Var}^{3} \cup \texttt{C}^{3},
      &\quad&\text{$\texttt{Var}^{3} < \texttt{Cls}^{3}$,}
      \\
      \texttt{Var}^{3} &= \bigcup_{i=1}^{n} \texttt{Var}^{3}_{i},
      &&\text{$\texttt{Var}^{3}_1 < \texttt{Var}^{3}_2 < \cdots < \texttt{Var}^{3}_n$,}
      \\
      \texttt{Cls}^{3} &= \bigcup_{j=1}^{m} \texttt{Cls}^{3}_{j},
      &&\text{$\texttt{Cls}^{3}_1 < \texttt{Cls}^{3}_2 < \cdots < \texttt{Cls}^{3}_m$.}
    \end{alignat*}
    where
    \begin{alignat*}{3}
      &\forall \leq i \leq n,\;&
      \texttt{Var}^{3}_{i} &= \texttt{Y}^{3}_{i,\LEFT} \cup \texttt{X}^{3}_{i} \cup \texttt{Y}^{3}_{i,\RIGHT},
      &\quad&\text{$\texttt{Y}^{3}_{i,\LEFT} < \texttt{X}^{3}_{i} < \texttt{Y}^{3}_{i,\RIGHT}$},
      \\
      &\forall \leq i \leq n,\;&
      \texttt{Y}^{3}_{i,\LEFT}
      &=
      \{\texttt{y}^{3}_{i,j,\LEFT} : 1 \leq j \leq N\},
      &&
      \\
      &\forall \leq i \leq n,\;&
      \texttt{X}^{3}_{i}
      &=
      \{\texttt{x}^{3}_{i,j,\FST}, \texttt{x}^{3}_{i,j,\SND} :
      \text{$x_{i} = c_{j}[2]$ or $\overline{x_{i}} = c_{j}[2]$}\},
      &&
      \\
      &\forall \leq i \leq n,\;&
      \texttt{Y}^{3}_{i,\RIGHT}
      &=
      \{\texttt{y}^{3}_{i,j,\RIGHT} : 1 \leq j \leq N\},
      &&
      \\
      &&\texttt{Cls}^{3}_{j}
      &=
      \{
      \texttt{l}^{3}_{j},
      \}
      \cup
      \{
      \texttt{ab}^{3}_{j,\LEFT},
      \texttt{cd}^{3}_{j,\LEFT},
      \texttt{ab}^{3}_{j,\RIGHT},
      \texttt{cd}^{3}_{j,\RIGHT}
      \}
      &&
    \end{alignat*}
    with
    $
    \texttt{ab}^{3}_{j,\LEFT} <
    \texttt{ab}^{3}_{j,\RIGHT} <
    \texttt{l}^{3}_{j} <
    \texttt{cd}^{3}_{j,\LEFT} <
    \texttt{cd}^{3}_{j,1,\RIGHT}
    $.
    All subsets
    $\texttt{Y}^{3}_{i,\LEFT}$, $\texttt{X}^{3}_{i}$ and $\texttt{Y}^{3}_{i,\RIGHT}$,
    are ordered according to the second coordinate index of the vertices
    (always written $j$ in the above definitions).

    For the edges $E^{3}$ of the linear graph $G^{3}$,
    define
    \begin{alignat*}{2}
      &&
      E^{3} &= E^{3}_{\texttt{Var}} \cup E^{3}_{\texttt{Var},\texttt{Cls}} \cup E^{3}_{\texttt{Cls}}
      \\
      &&
      E^{3}_{\texttt{Var}} &= \bigcup_{i=1}^{n} E^{3}_{\texttt{Var}, i}
      \\
      &\forall 1\leq i \leq n,\quad&
      E^{3}_{\texttt{Var}, 3} &= \bigcup_{k=1}^{N} (\texttt{y}^{3}_{i,k}, \texttt{y}^{3}_{i,N-k+1})
      \\
      &&
      E^{3}_{\texttt{Cls}} &= \bigcup_{j=1}^{m} E^{3}_{\texttt{Cls}, j}
      \\
      &\forall 1\leq j \leq m,\quad&
      E^{3}_{\texttt{Cls}, j} &=
      \{
      (\texttt{ab}^{3}_{j,\LEFT}, \texttt{ab}^{3}_{j,\RIGHT})
      \}\text{.}
    \end{alignat*}
    As for $E^{3}_{\texttt{Var},\texttt{Cls}}$,
    for every $\texttt{l}^{3}_{j} \in \texttt{Cls}^{3}_{j}$,
    we add the edge
    $(\texttt{x}^{3}_{i,j}, \texttt{l}^{3}_{j}) \in E^{3}_{\texttt{Var},\texttt{Cls}}$
    if $x_{i} = c_{j}[2]$ or $\overline{x_{i}} = c_{j}[2]$.
  % \end{mdframed}

  \bigskip

  We now claim that the $3$-CNF formula $\phi$ is satisfiable
  if and only if
  $G^{0}$ is $(G^{1}, G^{2}, G^{3})$-colorable.

  $(\Rightarrow)$
  Suppose that the $3$-CNF formula $\phi$ is satisfiable, and
  let $\gamma : X \to \{\FALSE, \TRUE\}$ be a satisfying assignment
  for $\phi$.
  Let $\COLOR_1$, $\COLOR_2$ and $\COLOR_3$ be three distinct colors.
  We define a mapping
  $\rho : V^{0} \to \{\COLOR_1, \COLOR_2, \COLOR_3\}$ as follows.

  \paragraph*{Gadgets $\texttt{Var}^{0}_{i}$}

  \begin{alignat*}{4}
    &\forall 1 \leq i \leq n, \forall 1 \leq j \leq N\;,
    &\quad
    \rho(\texttt{t}^{0}_{i, j, \LEFT}) &= \rho(\texttt{t}^{0}_{i, j, \RIGHT}) &=&
    \begin{cases}
      \COLOR_1 & \text{if $\gamma(x_i) = \TRUE$}\\
      \COLOR_2 & \text{if $\gamma(x_i) = \FALSE$}
    \end{cases}
    \\
    &\forall 1 \leq i \leq n, \forall 1 \leq j \leq N\;,
    &\quad
    \rho(\texttt{f}^{0}_{i, j, \LEFT}) &= \rho(\texttt{f}^{0}_{i, j, \RIGHT}) &=&
    \begin{cases}
      \COLOR_1 & \text{if $\gamma(x_i) = \FALSE$}\\
      \COLOR_2 & \text{if $\gamma(x_i) = \TRUE$}
    \end{cases}
    \\
    &\forall 1 \leq i \leq n, \forall 1 \leq j \leq N\;,
    &\quad
    \rho(\texttt{y}^{0}_{i, j, \LEFT}) &= \rho(\texttt{y}^{0}_{i, j, \RIGHT}) &=&
    \COLOR_3
    \\
    &\forall 1 \leq i \leq n, \forall 1 \leq j \leq \OCCURRENCE(x_i)\;,
    &\quad
    \rho(\texttt{x}^{0}_{i, 1, j}) &= \rho(\texttt{x}^{0}_{i, 3, j}) &=&
    \begin{cases}
      \COLOR_1 & \text{if $\gamma(x_i) = \TRUE$}\\
      \COLOR_2 & \text{if $\gamma(x_i) = \FALSE$}
    \end{cases}
    \\
    &\forall 1 \leq i \leq n, \forall 1 \leq j \leq \OCCURRENCE(x_i)\;,
    &\quad
    \rho(\neg\texttt{x}^{0}_{i, 1, j}) &= \rho(\neg\texttt{x}^{0}_{i, 3, j}) &=&
    \begin{cases}
      \COLOR_1 & \text{if $\gamma(x_i) = \FALSE$}\\
      \COLOR_2 & \text{if $\gamma(x_i) = \TRUE$}
    \end{cases}
    \\
  \end{alignat*}

  \paragraph*{Gadgets $\texttt{Cls}^{0}_{j}$}

  It can be checked that
  \begin{itemize}
    \item
    every vertex of $G^{0}$ is assigned exactly one color,
    \item
    every edge of $G^{0}$ connect two vertices with the same color,
    \item
    the subgraph of $G^{0}$ induced by color $\COLOR_1$
    (resp. $\COLOR_2$ and $\COLOR_3$) is isomorphic to $G^{1}$
    (resp. $G^{2}$ and $G^{3}$).
  \end{itemize}

  $(\Leftarrow)$
  Suppose now that $G^{0}$ is $(G^{1}, G^{2}, G^{3})$-colorable.
  We show how to construct a satifsying assignment
  $\gamma : X \to \{\FALSE, \TRUE\}$ for $\phi$.


  \begin{figure}
  \centering

  \begin{subfigure}[b]{\textwidth}
    \centering
    \begin{tikzpicture}
      [
      scale=.375,
      my vertex/.style={basic vertex,
      minimum size=1pt,
      inner sep=2pt,
      ball color=black!80},
      ]
      % vertices

      % p left
      \begin{scope}[]
        \fill [rounded corners, black!15] (-0.6,-0.6) -- (-0.6,0.6) --
        (3.6,0.6)   -- (3.6,-0.6) --
        cycle;
        \node [
        my vertex,
        label={[text depth=-1.75ex,label distance=0.25cm,rotate=-90]right:{\scriptsize $\texttt{f}^0_{i,1,\LEFT}$}}
        ] (T1l) at (0,0) {};
        \node [
        my vertex,
        label={[text depth=-1.75ex,label distance=0.25cm,rotate=-90]right:{\scriptsize $\texttt{f}^0_{i,2,\LEFT}$}}
        ] (T2l) at (1,0) {};
        \node [
        my vertex,
        label={[text depth=-1.75ex,label distance=0.25cm,rotate=-90]right:{\scriptsize $\texttt{f}^0_{i,N,\LEFT}$}}
        ] (TNl) at (3,0) {};
        \path [draw,dotted] (T2l) -- (TNl);
      \end{scope}

      % r left
      \begin{scope}[xshift=4.5cm]
        \fill [rounded corners, black!15] (-0.6,-0.6) -- (-0.6,0.6) --
        (3.6,0.6)   -- (3.6,-0.6) --
        cycle;
        \node [
        my vertex,
        label={[text depth=-1.75ex,label distance=0.25cm,rotate=-90]right:{\scriptsize $\texttt{y}^0_{i,1,\LEFT}$}}
        ] (Y1l) at (0,0) {};
        \node [
        my vertex,
        label={[text depth=-1.75ex,label distance=0.25cm,rotate=-90]right:{\scriptsize $\texttt{y}^0_{i,2,\LEFT}$}}
        ] (Y2l) at (1,0) {};
        \node [
        my vertex,
        label={[text depth=-1.75ex,label distance=0.25cm,rotate=-90]right:{\scriptsize $\texttt{y}^0_{i,N,\LEFT}$}}
        ] (YNl) at (3,0) {};
        \path [draw,dotted] (Y2l) -- (YNl);
      \end{scope}

      % x
      \begin{scope}[xshift=9cm]
        \fill [rounded corners, black!15] (-0.6,-0.6) -- (-0.6,0.6) --
        (3.6,0.6)   -- (3.6,-0.6) --
        cycle;
        \node [
        my vertex,
        label={[text depth=1.75ex,label distance=0.25cm,rotate=-90]left:{\scriptsize $\texttt{x}_{i,j_1}$}}
        ] (X1) at (0,0) {};
        \node [
        my vertex,
        label={[text depth=1.75ex,label distance=0.25cm,rotate=-90]left:{\scriptsize $\texttt{x}_{i,j_2}$}}
        ] (X2) at (1,0) {};
        \node [
        my vertex,
        label={[text depth=1.75ex,label distance=0.25cm,rotate=-90]left:{\scriptsize $\texttt{x}_{i,j_n}$}}
        ] (Xk) at (3,0) {};
        \path [draw,dotted] (X2) -- (Xk);
      \end{scope}

      % q left
      \begin{scope}[xshift=13.5cm]
        \fill [rounded corners, black!15] (-0.6,-0.6) -- (-0.6,0.6) --
        (3.6,0.6)   -- (3.6,-0.6) --
        cycle;
        \node [
        my vertex,
        label={[text depth=-1.75ex,label distance=0.25cm,rotate=-90]right:{\scriptsize $\texttt{f}^0_{i,1,\LEFT}$}}
        ] (F1l) at (0,0) {};
        \node [
        my vertex,
        label={[text depth=-1.75ex,label distance=0.25cm,rotate=-90]right:{\scriptsize $\texttt{f}^0_{i,2,\LEFT}$}}
        ] (F2l) at (1,0) {};
        \node [
        my vertex,
        label={[text depth=-1.75ex,label distance=0.25cm,rotate=-90]right:{\scriptsize $\texttt{f}^0_{i,N,\LEFT}$}}
        ] (FNl) at (3,0) {};
        \path [draw,dotted] (F2l) -- (FNl);
      \end{scope}

      % p right
      \begin{scope}[xshift=18cm]
        \fill [rounded corners, black!15] (-0.6,-0.6) -- (-0.6,0.6) --
        (3.6,0.6)   -- (3.6,-0.6) --
        cycle;
        \node [
        my vertex,
        label={[text depth=-1.75ex,label distance=0.25cm,rotate=-90]right:{\scriptsize $\texttt{f}^0_{i,1,\RIGHT}$}}
        ] (T1r) at (0,0) {};
        \node [
        my vertex,
        label={[text depth=-1.75ex,label distance=0.25cm,rotate=-90]right:{\scriptsize $\texttt{f}^0_{i,2,\RIGHT}$}}
        ] (T2r) at (1,0) {};
        \node [
        my vertex,
        label={[text depth=-1.75ex,label distance=0.25cm,rotate=-90]right:{\scriptsize $\texttt{f}^0_{i,N,\RIGHT}$}}
        ] (TNr) at (3,0) {};
        \path [draw,dotted] (T2r) -- (TNr);
      \end{scope}

      % neg x
      \begin{scope}[xshift=22.5cm]
        \fill [rounded corners, black!15] (-0.6,-0.6) -- (-0.6,0.6) --
        (3.6,0.6)   -- (3.6,-0.6) --
        cycle;
        \node [
        my vertex,
        label={[text depth=1.75ex,label distance=0.25cm,rotate=-90]left:{\scriptsize $\neg\texttt{x}_{i,j_1}$}}
        ] (nX1) at (0,0) {};
        \node [
        my vertex,
        label={[text depth=1.75ex,label distance=0.25cm,rotate=-90]left:{\scriptsize $\neg\texttt{x}_{i,j_2}$}}
        ] (nX2) at (1,0) {};
        \node [
        my vertex,
        label={[text depth=1.75ex,label distance=0.25cm,rotate=-90]left:{\scriptsize $\neg\texttt{x}_{i,j_n}$}}
        ] (nXk) at (3,0) {};
        \path [draw,dotted] (nX2) -- (nXk);
      \end{scope}

      % r right
      \begin{scope}[xshift=27cm]
        \fill [rounded corners, black!15] (-0.6,-0.6) -- (-0.6,0.6) --
        (3.6,0.6)   -- (3.6,-0.6) --
        cycle;
        \node [
        my vertex,
        label={[text depth=-1.75ex,label distance=0.25cm,rotate=-90]right:{\scriptsize $\texttt{y}^0_{i,1,\RIGHT}$}}
        ] (Y1r) at (0,0) {};
        \node [
        my vertex,
        label={[text depth=-1.75ex,label distance=0.25cm,rotate=-90]right:{\scriptsize $\texttt{y}^0_{i,2,\RIGHT}$}}
        ] (Y2r) at (1,0) {};
        \node [
        my vertex,
        label={[text depth=-1.75ex,label distance=0.25cm,rotate=-90]right:{\scriptsize $\texttt{y}^0_{i,N,\RIGHT}$}}
        ] (YNr) at (3,0) {};
        \path [draw,dotted] (Y2r) -- (YNr);
      \end{scope}

      % q right
      \begin{scope}[xshift=31.5cm]
        \fill [rounded corners, black!15] (-0.6,-0.6) -- (-0.6,0.6) --
        (3.6,0.6)   -- (3.6,-0.6) --
        cycle;
        \node [
        my vertex,
        label={[text depth=-1.75ex,label distance=0.25cm,rotate=-90]right:{\scriptsize $\texttt{f}^0_{i,1,\RIGHT}$}}
        ] (F1r) at (0,0) {};
        \node [
        my vertex,
        label={[text depth=-1.75ex,label distance=0.25cm,rotate=-90]right:{\scriptsize $\texttt{f}^0_{i,2,\RIGHT}$}}
        ] (F2r) at (1,0) {};
        \node [
        my vertex,
        label={[text depth=-1.75ex,label distance=0.25cm,rotate=-90]right:{\scriptsize $\texttt{f}^0_{i,N,\RIGHT}$}}
        ] (FNr) at (3,0) {};
        \path [draw,dotted] (F2r) -- (FNr);
      \end{scope}

      % edges
      \path [edge,color=blue]
      (T1l.north) -- ++(0,5.5) -- ($(TNr.north)+(0,5.5)$) node [pos=0.1,above] {$\COLOR_1$} -- (TNr.north);
      \path [edge,color=blue,path fading=east]
      (T2l.north) -- ++(0,5) --  ($(T2r.north)+(0,5)$);
      \path [edge,color=blue,path fading=west]
      (T2r.north) -- ++(0,4) -- ($(TNl.north)+(0,4)$);
      \path [edge,color=blue]
      (TNl.north) -- ($(TNl.north)+(0,3.5)$) -- ($(T1r.north)+(0,3.5)$) -- (T1r.north);

      \path [edge,color=green]
      (F1l.north) -- ++(0,13) -- ($(FNr.north)+(0,13)$) node [pos=0.1,above] {$\COLOR_2$} -- (FNr.north);
      \path [edge,color=green,path fading=east]
      (F2l.north) -- ++(0,12.5) --  ($(F2r.north)+(0,12.5)$);
      \path [edge,color=green,path fading=west]
      (F2r.north) -- ++(0,11.5) -- ($(FNl.north)+(0,11.5)$);
      \path [edge,color=green]
      (FNl.north) -- ($(FNl.north)+(0,10.5)$) -- ($(F1r.north)+(0,10.5)$) -- (F1r.north);

      \path [edge,color=red]
      (Y1l.north) -- ++(0,9) -- ($(YNr.north)+(0,9)$) node [pos=0.1,above] {$\COLOR_3$} -- (YNr.north);
      \path [edge,color=red,path fading=east]
      (Y2l.north) -- ++(0,8.5) --  ($(Y2r.north)+(0,8.5)$);
      \path [edge,color=red,path fading=west]
      (Y2r.north) -- ++(0,7.5) -- ($(YNl.north)+(0,7.5)$);
      \path [edge,color=red]
      (YNl.north) -- ($(YNl.north)+(0,7)$) -- ($(Y1r.north)+(0,7)$) -- (Y1r.north);

      \path [edge,side by side={red}{blue},path fading=east] (X1.south) -- ++(0,-5) -- ++(18,0);
      \path [edge,side by side={red}{blue},path fading=east] (X2.south) -- ++(0,-5.5) -- ++(18,0);
      \path [edge,side by side={red}{blue},path fading=east] (Xk.south) -- ++(0,-6.5) -- ++(17,0);
      \node [text width=2cm] at ($(Xk.south)+(3,-4)$) {$\color{blue}{\COLOR_1} \color{black}{\vee} \color{red}{\COLOR_3}$};

      \path [edge,side by side={red}{green},path fading=east] (nX1.south) -- ++(0,-7) -- ++(7.5,0);
      \path [edge,side by side={red}{green},path fading=east] (nX2.south) -- ++(0,-7.5) -- ++(7.5,0);
      \path [edge,side by side={red}{green},path fading=east] (nXk.south)  -- ++(0,-8.5) -- ++(6.5,0);
      \node [text width=2cm] at ($(nXk.south)+(3,-4)$) {$\color{green}{\COLOR_2} \color{black}{\vee} \color{red}{\COLOR_3}$};

      %\node [text width=2cm,anchor=west] at (31,-7) {\scriptsize Towards the clause gadgets};
    \end{tikzpicture}
    \caption{Setting $\rho(x_i) = \text{true}$ in variable gadget $\texttt{Var}^{0}_{i}$.}
  \end{subfigure}

  \vspace{.5cm}

  \begin{subfigure}[b]{\textwidth}
    \centering
    \begin{tikzpicture}
      [
      scale=.375,
      my vertex/.style={basic vertex,
      minimum size=1pt,
      inner sep=2pt,
      ball color=black!80},
      ]
      % vertices

      % p left
      \begin{scope}[]
        \fill [rounded corners, black!15] (-0.6,-0.6) -- (-0.6,0.6) --
        (3.6,0.6)   -- (3.6,-0.6) --
        cycle;
        \node [
        my vertex,
        label={[text depth=-1.75ex,label distance=0.25cm,rotate=-90]right:{\scriptsize $\texttt{f}^0_{i,1,\LEFT}$}}
        ] (T1l) at (0,0) {};
        \node [
        my vertex,
        label={[text depth=-1.75ex,label distance=0.25cm,rotate=-90]right:{\scriptsize $\texttt{f}^0_{i,2,\LEFT}$}}
        ] (T2l) at (1,0) {};
        \node [
        my vertex,
        label={[text depth=-1.75ex,label distance=0.25cm,rotate=-90]right:{\scriptsize $\texttt{f}^0_{i,N,\LEFT}$}}
        ] (TNl) at (3,0) {};
        \path [draw,dotted] (T2l) -- (TNl);
      \end{scope}

      % r left
      \begin{scope}[xshift=4.5cm]
        \fill [rounded corners, black!15] (-0.6,-0.6) -- (-0.6,0.6) --
        (3.6,0.6)   -- (3.6,-0.6) --
        cycle;
        \node [
        my vertex,
        label={[text depth=-1.75ex,label distance=0.25cm,rotate=-90]right:{\scriptsize $\texttt{y}^0_{i,1,\LEFT}$}}
        ] (Y1l) at (0,0) {};
        \node [
        my vertex,
        label={[text depth=-1.75ex,label distance=0.25cm,rotate=-90]right:{\scriptsize $\texttt{y}^0_{i,2,\LEFT}$}}
        ] (Y2l) at (1,0) {};
        \node [
        my vertex,
        label={[text depth=-1.75ex,label distance=0.25cm,rotate=-90]right:{\scriptsize $\texttt{y}^0_{i,N,\LEFT}$}}
        ] (YNl) at (3,0) {};
        \path [draw,dotted] (Y2l) -- (YNl);
      \end{scope}

      % x
      \begin{scope}[xshift=9cm]
        \fill [rounded corners, black!15] (-0.6,-0.6) -- (-0.6,0.6) --
        (3.6,0.6)   -- (3.6,-0.6) --
        cycle;
        \node [
        my vertex,
        label={[text depth=1.75ex,label distance=0.25cm,rotate=-90]left:{\scriptsize $\texttt{x}_{i,j_1}$}}
        ] (X1) at (0,0) {};
        \node [
        my vertex,
        label={[text depth=1.75ex,label distance=0.25cm,rotate=-90]left:{\scriptsize $\texttt{x}_{i,j_2}$}}
        ] (X2) at (1,0) {};
        \node [
        my vertex,
        label={[text depth=1.75ex,label distance=0.25cm,rotate=-90]left:{\scriptsize $\texttt{x}_{i,j_n}$}}
        ] (Xk) at (3,0) {};
        \path [draw,dotted] (X2) -- (Xk);
      \end{scope}

      % q left
      \begin{scope}[xshift=13.5cm]
        \fill [rounded corners, black!15] (-0.6,-0.6) -- (-0.6,0.6) --
        (3.6,0.6)   -- (3.6,-0.6) --
        cycle;
        \node [
        my vertex,
        label={[text depth=-1.75ex,label distance=0.25cm,rotate=-90]right:{\scriptsize $\texttt{f}^0_{i,1,\LEFT}$}}
        ] (F1l) at (0,0) {};
        \node [
        my vertex,
        label={[text depth=-1.75ex,label distance=0.25cm,rotate=-90]right:{\scriptsize $\texttt{f}^0_{i,2,\LEFT}$}}
        ] (F2l) at (1,0) {};
        \node [
        my vertex,
        label={[text depth=-1.75ex,label distance=0.25cm,rotate=-90]right:{\scriptsize $\texttt{f}^0_{i,N,\LEFT}$}}
        ] (FNl) at (3,0) {};
        \path [draw,dotted] (F2l) -- (FNl);
      \end{scope}

      % p right
      \begin{scope}[xshift=18cm]
        \fill [rounded corners, black!15] (-0.6,-0.6) -- (-0.6,0.6) --
        (3.6,0.6)   -- (3.6,-0.6) --
        cycle;
        \node [
        my vertex,
        label={[text depth=-1.75ex,label distance=0.25cm,rotate=-90]right:{\scriptsize $\texttt{f}^0_{i,1,\RIGHT}$}}
        ] (T1r) at (0,0) {};
        \node [
        my vertex,
        label={[text depth=-1.75ex,label distance=0.25cm,rotate=-90]right:{\scriptsize $\texttt{f}^0_{i,2,\RIGHT}$}}
        ] (T2r) at (1,0) {};
        \node [
        my vertex,
        label={[text depth=-1.75ex,label distance=0.25cm,rotate=-90]right:{\scriptsize $\texttt{f}^0_{i,N,\RIGHT}$}}
        ] (TNr) at (3,0) {};
        \path [draw,dotted] (T2r) -- (TNr);
      \end{scope}


      % neg x
      \begin{scope}[xshift=22.5cm]
        \fill [rounded corners, black!15] (-0.6,-0.6) -- (-0.6,0.6) --
        (3.6,0.6)   -- (3.6,-0.6) --
        cycle;
        \node [
        my vertex,
        label={[text depth=1.75ex,label distance=0.25cm,rotate=-90]left:{\scriptsize $\neg\texttt{x}_{i,j_1}$}}
        ] (nX1) at (0,0) {};
        \node [
        my vertex,
        label={[text depth=1.75ex,label distance=0.25cm,rotate=-90]left:{\scriptsize $\neg\texttt{x}_{i,j_2}$}}
        ] (nX2) at (1,0) {};
        \node [
        my vertex,
        label={[text depth=1.75ex,label distance=0.25cm,rotate=-90]left:{\scriptsize $\neg\texttt{x}_{i,j_n}$}}
        ] (nXk) at (3,0) {};
        \path [draw,dotted] (nX2) -- (nXk);
      \end{scope}

      % r right
      \begin{scope}[xshift=27cm]
        \fill [rounded corners, black!15] (-0.6,-0.6) -- (-0.6,0.6) --
        (3.6,0.6)   -- (3.6,-0.6) --
        cycle;
        \node [
        my vertex,
        label={[text depth=-1.75ex,label distance=0.25cm,rotate=-90]right:{\scriptsize $\texttt{y}^0_{i,1,\RIGHT}$}}
        ] (Y1r) at (0,0) {};
        \node [
        my vertex,
        label={[text depth=-1.75ex,label distance=0.25cm,rotate=-90]right:{\scriptsize $\texttt{y}^0_{i,2,\RIGHT}$}}
        ] (Y2r) at (1,0) {};
        \node [
        my vertex,
        label={[text depth=-1.75ex,label distance=0.25cm,rotate=-90]right:{\scriptsize $\texttt{y}^0_{i,N,\RIGHT}$}}
        ] (YNr) at (3,0) {};
        \path [draw,dotted] (Y2r) -- (YNr);
      \end{scope}

      % q right
      \begin{scope}[xshift=31.5cm]
        \fill [rounded corners, black!15] (-0.6,-0.6) -- (-0.6,0.6) --
        (3.6,0.6)   -- (3.6,-0.6) --
        cycle;
        \node [
        my vertex,
        label={[text depth=-1.75ex,label distance=0.25cm,rotate=-90]right:{\scriptsize $\texttt{f}^0_{i,1,\RIGHT}$}}
        ] (F1r) at (0,0) {};
        \node [
        my vertex,
        label={[text depth=-1.75ex,label distance=0.25cm,rotate=-90]right:{\scriptsize $\texttt{f}^0_{i,2,\RIGHT}$}}
        ] (F2r) at (1,0) {};
        \node [
        my vertex,
        label={[text depth=-1.75ex,label distance=0.25cm,rotate=-90]right:{\scriptsize $\texttt{f}^0_{i,N,\RIGHT}$}}
        ] (FNr) at (3,0) {};
        \path [draw,dotted] (F2r) -- (FNr);
      \end{scope}

      % edges
      \path [edge,color=green]
      (T1l.north) -- ++(0,5.5) -- ($(TNr.north)+(0,5.5)$) node [pos=0.1,above] {$\COLOR_2$} -- (TNr.north);
      \path [edge,color=green,path fading=east]
      (T2l.north) -- ++(0,5) --  ($(T2r.north)+(0,5)$);
      \path [edge,color=green,path fading=west]
      (T2r.north) -- ++(0,4) -- ($(TNl.north)+(0,4)$);
      \path [edge,color=green]
      (TNl.north) -- ($(TNl.north)+(0,3.5)$) -- ($(T1r.north)+(0,3.5)$) -- (T1r.north);

      \path [edge,color=blue]
      (F1l.north) -- ++(0,13) -- ($(FNr.north)+(0,13)$) node [pos=0.1,above] {$\COLOR_1$} -- (FNr.north);
      \path [edge,color=blue,path fading=east]
      (F2l.north) -- ++(0,12.5) --  ($(F2r.north)+(0,12.5)$);
      \path [edge,color=blue,path fading=west]
      (F2r.north) -- ++(0,11.5) -- ($(FNl.north)+(0,11.5)$);
      \path [edge,color=blue]
      (FNl.north) -- ($(FNl.north)+(0,10.5)$) -- ($(F1r.north)+(0,10.5)$) -- (F1r.north);

      \path [edge,color=red]
      (Y1l.north) -- ++(0,9) -- ($(YNr.north)+(0,9)$) node [pos=0.1,above] {$\COLOR_3$} -- (YNr.north);
      \path [edge,color=red,path fading=east]
      (Y2l.north) -- ++(0,8.5) --  ($(Y2r.north)+(0,8.5)$);
      \path [edge,color=red,path fading=west]
      (Y2r.north) -- ++(0,7.5) -- ($(YNl.north)+(0,7.5)$);
      \path [edge,color=red]
      (YNl.north) -- ($(YNl.north)+(0,7)$) -- ($(Y1r.north)+(0,7)$) -- (Y1r.north);

      \path [edge,side by side={red}{green},path fading=east] (X1.south) -- ++(0,-5) -- ++(18,0);
      \path [edge,side by side={red}{green},path fading=east] (X2.south) -- ++(0,-5.5) -- ++(18,0);
      \path [edge,side by side={red}{green},path fading=east] (Xk.south) -- ++(0,-6.5) -- ++(17,0);
      \node [text width=2cm] at ($(Xk.south)+(3,-4)$) {$\color{green}{\COLOR_2} \color{black}{\vee} \color{red}{\COLOR_3}$};

      \path [edge,side by side={red}{blue},path fading=east] (nX1.south) -- ++(0,-7) -- ++(7.5,0);
      \path [edge,side by side={red}{blue},path fading=east] (nX2.south) -- ++(0,-7.5) -- ++(7.5,0);
      \path [edge,side by side={red}{blue},path fading=east] (nXk.south) -- ++(0,-8.5) -- ++(6.5,0);
      \node [text width=2cm] at ($(nXk.south)+(3,-4)$) {$\color{blue}{\COLOR_1} \color{black}{\vee} \color{red}{\COLOR_3}$};

      %\node [text width=2cm,anchor=west] at (31,-7) {\scriptsize Towards the clause gadgets};
    \end{tikzpicture}
    \caption{Setting $\rho(x_i) = \text{false}$ in variable gadget $\texttt{Var}^{0}_{i}$.}
  \end{subfigure}
\end{figure}


  
\begin{figure}
  \centering

  \begin{subfigure}[b]{\textwidth}
    \centering
    \begin{tikzpicture}
      [
      scale=.22,,,
      my vertex/.style={basic vertex,
      minimum size=1pt,
      inner sep=2pt,
      ball color=black!80},
      ]

      % vertices
      \foreach \i/\o/\l [evaluate=\i as \x using \i*3.5] in
      {
      1/above/\large\texttt{l}^0_{j,1,\TRUE},
      2/below/\texttt{a}^0_{j,\LEFT},
      3/above/\texttt{l}^0_{j,1,\FALSE},
      4/below/\texttt{b}^0_{j,\LEFT},
      5/below/\texttt{a}^0_{j,\RIGHT},
      6/above/\texttt{l}^0_{j,2,\FALSE,\FST},
      7/below/\texttt{b}^0_{j,\RIGHT},
      8/above/\texttt{l}^0_{j,2,\TRUE},
      9/below/\texttt{c}^0_{j,\LEFT},
      10/above/\texttt{l}^0_{j,2,\FALSE,\SND},
      11/below/\texttt{d}^0_{j,\LEFT},
      12/below/\texttt{c}^0_{j,\RIGHT},
      13/above/\texttt{l}^0_{j,3,\FALSE},
      14/below/\texttt{d}^0_{j,\RIGHT},
      15/above/\texttt{l}^0_{j,3,\TRUE}
      }
      {
      \node [my vertex,label=\o:{\scriptsize $\l$}] (V\i) at (\x,0) {};
      }

      % edges
      \path [edge,color=blue] (V2.north) -- ++ (0,4) --
      ($(V5.north)+(0,4)$) node [pos=0.4,above] {$\COLOR_1$} -- (V5.north);

      \path [edge,color=red] (V4.north) -- ++ (0,5) --
      ($(V7.north)+(0,5)$) node [pos=0.6,above] {$\COLOR_3$} -- (V7.north);

      \path [edge,color=blue] (V9.north) -- ++ (0,5) --
      ($(V12.north)+(0,5)$) node [pos=0.4,above] {$\COLOR_1$} -- (V12.north);

      \path [edge,color=red] (V11.north) -- ++ (0,4) --
      ($(V14.north)+(0,4)$) node [pos=0.6,above] {$\COLOR_3$} -- (V14.north);

      % incoming edges


      \draw [edge,color=green,path fading=west] (V3) -- node [pos=0.75,anchor=west] {$\COLOR_2$}
      ++(0,-6) -- +(-10.5,0);

      \draw [edge,color=blue,path fading=west] (V6) -- node [pos=0.75,anchor=west] {$\COLOR_1$}
      ++(0,-7) -- +(-21.5,0);

      \draw [edge,color=green,path fading=west] (V8) -- node [pos=0.75,anchor=west] {$\COLOR_2$}
      ++(0,-8)  -- +(-29,0);

      \draw [edge,color=red,path fading=west] (V10) -- node [pos=0.75,anchor=west] {$\COLOR_3$}
      ++(0,-9) -- +(-36.5,0);

      \draw [edge,color=blue,path fading=west] (V13) -- node [pos=0.75,anchor=west] {$\COLOR_1$}
      ++(0,-10) -- +(-47.5,0);

      \draw [edge,color=green,path fading=west] (V15) -- node [pos=0.75,anchor=west] {$\COLOR_2$}
      ++(0,-11) -- +(-55,0);

      \draw [edge,color=blue,path fading=west] (V1) -- node [pos=0.75,anchor=west] {$\COLOR_1$}
      ++(0,-5) -- +(-3,0);
      %\node [text width=2.5cm,anchor=west] at (-15.5,-8.5) {\scriptsize Towards the variable gadgets};

    \end{tikzpicture}
    \caption{%
    Clause gadget $\texttt{Cls}^{0}_{j}$ for clause $(\ell_{j,1} \vee \ell_{j,2} \vee \ell_{j,3})$
    in case the literals $\ell_{j,1}$, $\ell_{j,2}$ and $\ell_{j,3}$ evaluate to true, false and false,
    respectivelly, according to the satisfying assignment $f$.
    }% end caption  \end{subfigure}
  \end{subfigure}

  \vspace{.5cm}

  \begin{subfigure}[b]{\textwidth}
    \centering
    \begin{tikzpicture}
      [
      scale=.22,,,
      my vertex/.style={basic vertex,
      minimum size=1pt,
      inner sep=2pt,
      ball color=black!80},
      ]

      % vertices
      \foreach \i/\o/\l [evaluate=\i as \x using \i*3.5] in
      {
      1/above/\texttt{l}^0_{j,1,\TRUE},
      2/below/\texttt{a}^0_{j,\LEFT},
      3/above/\texttt{l}^0_{j,1,\FALSE},
      4/below/\texttt{b}^0_{j,\LEFT},
      5/below/\texttt{a}^0_{j,\RIGHT},
      6/above/\texttt{l}^0_{j,2,\FALSE,\FST},
      7/below/\texttt{b}^0_{j,\RIGHT},
      8/above/\texttt{l}^0_{j,2,\TRUE},
      9/below/\texttt{c}^0_{j,\LEFT},
      10/above/\texttt{l}^0_{j,2,\FALSE,\SND},
      11/below/\texttt{d}^0_{j,\LEFT},
      12/below/\texttt{c}^0_{j,\RIGHT},
      13/above/\texttt{l}^0_{j,3,\FALSE},
      14/below/\texttt{d}^0_{j,\RIGHT},
      15/above/\texttt{l}^0_{j,3,\TRUE}
      }
      {
      \node [my vertex,label=\o:{\scriptsize $\l$}] (V\i) at (\x,0) {};
      }

      % edges
      \path [edge,color=red] (V2.north) -- ++ (0,4) --
      ($(V5.north)+(0,4)$) node [pos=0.4,above] {$\COLOR_3$}  -- (V5.north);

      \path [edge,color=blue] (V4.north) -- ++ (0,5) --
      ($(V7.north)+(0,5)$) node [pos=0.6,above] {$\COLOR_1$} -- (V7.north);

      \path [edge,color=blue] (V9.north) -- ++ (0,5) --
      ($(V12.north)+(0,5)$) node [pos=0.4,above] {$\COLOR_1$} -- (V12.north);

      \path [edge,color=red] (V11.north) -- ++ (0,4) --
      ($(V14.north)+(0,4)$) node [pos=0.6,above] {$\COLOR_3$} -- (V14.north);

      % incoming edges
      \draw [edge,color=green,path fading=west] (V1) -- node [pos=0.75,anchor=west] {$\COLOR_2$}
      ++(0,-5) -- +(-3,0) node {};

      \draw [edge,color=blue,path fading=west] (V3) -- node [pos=0.75,anchor=west] {$\COLOR_1$}
      ++(0,-6) -- +(-10.5,0) node {};

      \draw [edge,color=green,path fading=west] (V6)  -- node [pos=0.75,anchor=west] {$\COLOR_2$}
      ++(0,-7)  -- +(-21.5,0) node {};

      \draw [edge,color=blue,path fading=west] (V8)  -- node [pos=0.75,anchor=west] {$\COLOR_1$}
      ++(0,-8)  -- +(-29,0) node {};

      \draw [edge,color=red,path fading=west] (V10) -- node [pos=0.75,anchor=west] {$\COLOR_3$}
      ++(0,-9) -- +(-36.5,0) node {};

      \draw [edge,color=blue,path fading=west] (V13) -- node [pos=0.75,anchor=west] {$\COLOR_1$}
      ++(0,-10) -- +(-47.5,0) node {};

      \draw [edge,color=green,path fading=west] (V15) -- node [pos=0.75,anchor=west] {$\COLOR_2$}
      ++(0,-11) -- +(-55,0) node {};

      %\node [text width=2.5cm,anchor=west] at (-15.5,-8.5) {\scriptsize Towards the variable gadgets};

    \end{tikzpicture}
    \caption{%
    Clause gadget $\texttt{Cls}^{0}_{j}$ for clause $(\ell_{j,1} \vee \ell_{j,2} \vee \ell_{j,3})$
    in case the literals $\ell_{j,1}$, $\ell_{j,2}$ and $\ell_{j,3}$ evaluate to false, true and false,
    respectivelly, according to the satisfying assignment $f$.
    }% end caption  \end{subfigure}
  \end{subfigure}

  \vspace{.5cm}

  \begin{subfigure}[b]{\textwidth}
    \centering
    \begin{tikzpicture}
      [
      scale=.22,,,
      my vertex/.style={basic vertex,
      minimum size=1pt,
      inner sep=2pt,
      ball color=black!80},
      ]

      % vertices
      \foreach \i/\o/\l [evaluate=\i as \x using \i*3.5] in
      {
      1/above/\texttt{l}^0_{j,1,\TRUE},
      2/below/\texttt{a}^0_{j,\LEFT},
      3/above/\texttt{l}^0_{j,1,\FALSE},
      4/below/\texttt{b}^0_{j,\LEFT},
      5/below/\texttt{a}^0_{j,\RIGHT},
      6/above/\texttt{l}^0_{j,2,\FALSE,\FST},
      7/below/\texttt{b}^0_{j,\RIGHT},
      8/above/\texttt{l}^0_{j,2,\TRUE},
      9/below/\texttt{c}^0_{j,\LEFT},
      10/above/\texttt{l}^0_{j,2,\FALSE,\SND},
      11/below/\texttt{d}^0_{j,\LEFT},
      12/below/\texttt{c}^0_{j,\RIGHT},
      13/above/\texttt{l}^0_{j,3,\FALSE},
      14/below/\texttt{d}^0_{j,\RIGHT},
      15/above/\texttt{l}^0_{j,3,\TRUE}
      }
      {
      \node [my vertex,label=\o:{\scriptsize $\l$}] (V\i) at (\x,0) {};
      }

      % edges
      \path [edge,color=red] (V2.north) -- ++ (0,4) --
      ($(V5.north)+(0,4)$) node [pos=0.4,above] {$\COLOR_3$}   -- (V5.north);

      \path [edge,color=blue] (V4.north) -- ++ (0,5) --
      ($(V7.north)+(0,5)$) node [pos=0.6,above] {$\COLOR_1$} -- (V7.north);

      \path [edge,color=red] (V9.north) -- ++ (0,5) --
      ($(V12.north)+(0,5)$) node [pos=0.4,above] {$\COLOR_3$} -- (V12.north);

      \path [edge,color=blue] (V11.north) -- ++ (0,4) --
      ($(V14.north)+(0,4)$) node [pos=0.6,above] {$\COLOR_1$} -- (V14.north);

      % incoming edges
      \draw [edge,color=green,path fading=west] (V1) -- node [pos=0.75,anchor=west] {$\COLOR_2$}
      ++(0,-5) -- +(-3,0) node {};

      \draw [edge,color=blue,path fading=west] (V3) -- node [pos=0.75,anchor=west] {$\COLOR_1$}
      ++(0,-6) -- +(-10.5,0) node {};

      \draw [edge,color=red,path fading=west] (V6)  -- node [pos=0.75,anchor=west] {$\COLOR_3$}
      ++(0,-7)  -- +(-21.5,0) node {};

      \draw [edge,color=green,path fading=west] (V8)  -- node [pos=0.75,anchor=west] {$\COLOR_2$}
      ++(0,-8)  -- +(-29,0) node {};

      \draw [edge,color=blue,path fading=west] (V10) -- node [pos=0.75,anchor=west] {$\COLOR_1$}
      ++(0,-9) -- +(-36.5,0) node {};

      \draw [edge,color=green,path fading=west] (V13) -- node [pos=0.75,anchor=west] {$\COLOR_2$}
      ++(0,-10) -- +(-47.5,0) node {};

      \draw [edge,color=blue,path fading=west] (V15) -- node [pos=0.75,anchor=west] {$\COLOR_1$}
      ++(0,-11) -- +(-55,0) node {};

      %\node [text width=2.5cm,anchor=west] at (-15.5,-8.5) {\scriptsize Towards the variable gadgets};

    \end{tikzpicture}
    \caption{%
    Clause gadget $\texttt{Cls}^{0}_{j}$ for clause $(\ell_{j,1} \vee \ell_{j,2} \vee \ell_{j,3})$
    in case the literals $\ell_{j,1}$, $\ell_{j,2}$ and $\ell_{j,3}$ evaluate to false, false and true,
    respectivelly, according to the satisfying assignment $f$.
    }% end caption  \end{subfigure}
  \end{subfigure}

  \vspace{.5cm}

  \begin{subfigure}[b]{\textwidth}
    \centering
    \begin{tikzpicture}
      [
      scale=.22,,,
      my vertex/.style={basic vertex,
      minimum size=1pt,
      inner sep=2pt,
      ball color=black!80},
      ]

      % vertices
      \foreach \i/\o/\l [evaluate=\i as \x using \i*3.5] in
      {
      1/above/\texttt{l}^0_{j,1,\TRUE},
      2/below/\texttt{a}^0_{j,\LEFT},
      3/above/\texttt{l}^0_{j,1,\FALSE},
      4/below/\texttt{b}^0_{j,\LEFT},
      5/below/\texttt{a}^0_{j,\RIGHT},
      6/above/\texttt{l}^0_{j,2,\FALSE,\FST},
      7/below/\texttt{b}^0_{j,\RIGHT},
      8/above/\texttt{l}^0_{j,2,\TRUE},
      9/below/\texttt{c}^0_{j,\LEFT},
      10/above/\texttt{l}^0_{j,2,\FALSE,\SND},
      11/below/\texttt{d}^0_{j,\LEFT},
      12/below/\texttt{c}^0_{j,\RIGHT},
      13/above/\texttt{l}^0_{j,3,\FALSE},
      14/below/\texttt{d}^0_{j,\RIGHT},
      15/above/\texttt{l}^0_{j,3,\TRUE}
      }
      {
      \node [my vertex,label=\o:{\scriptsize $\l$}] (V\i) at (\x,0) {};
      }

      % edges
      \path [edge,color=red] (V2.north) -- ++ (0,4) --
      ($(V5.north)+(0,4)$) node [pos=0.4,above] {$\COLOR_3$} -- (V5.north);

      \path [edge,color=blue] (V4.north) -- ++ (0,5) --
      ($(V7.north)+(0,5)$) node [pos=0.6,above] {$\COLOR_1$} -- (V7.north);

      \path [edge,color=blue] (V9.north) -- ++ (0,5) --
      ($(V12.north)+(0,5)$) node [pos=0.4,above] {$\COLOR_1$} -- (V12.north);

      \path [edge,color=red] (V11.north) -- ++ (0,4) --
      ($(V14.north)+(0,4)$) node [pos=0.6,above] {$\COLOR_3$} -- (V14.north);

      % incoming edges
      \draw [edge,color=blue,path fading=west] (V1) -- node [pos=0.75,anchor=west] {$\COLOR_1$}
      ++(0,-5) -- +(-3,0) node {};

      \draw [edge,color=green,path fading=west] (V3) -- node [pos=0.75,anchor=west] {$\COLOR_2$}
      ++(0,-6) -- +(-10.5,0) node {};

      \draw [edge,color=green,path fading=west] (V6)  -- node [pos=0.75,anchor=west] {$\COLOR_2$}
      ++(0,-7)  -- +(-21.5,0) node {};

      \draw [edge,color=blue,path fading=west] (V8)  -- node [pos=0.75,anchor=west] {$\COLOR_1$}
      ++(0,-8)  -- +(-29,0) node {};

      \draw [edge,color=red,path fading=west] (V10) --  node [pos=0.75,anchor=west] {$\COLOR_3$}
      ++(0,-9) -- +(-36.5,0) node {};

      \draw [edge,color=blue,path fading=west] (V13) -- node [pos=0.75,anchor=west] {$\COLOR_1$}
      ++(0,-10) -- +(-47.5,0) node {};

      \draw [edge,color=green,path fading=west] (V15) -- node [pos=0.75,anchor=west] {$\COLOR_2$}
      ++(0,-11) -- +(-55,0) node {};

      %\node [text width=2.5cm,anchor=west] at (-15.5,-8.5) {\scriptsize Towards the variable gadgets};

    \end{tikzpicture}
    \caption{%
    Clause gadget $\texttt{Cls}^{0}_{j}$ for clause $(\ell_{j,1} \vee \ell_{j,2} \vee \ell_{j,3})$
    in case the literals $\ell_{j,1}$, $\ell_{j,2}$ and $\ell_{j,3}$ evaluate to true, true and false,
    respectivelly, according to the satisfying assignment $f$.
    }% end caption  \end{subfigure}
  \end{subfigure}

\end{figure}

\begin{figure}\ContinuedFloat

  \begin{subfigure}[b]{\textwidth}
    \centering
    \begin{tikzpicture}
      [
      scale=.22,,,
      my vertex/.style={basic vertex,
      minimum size=1pt,
      inner sep=2pt,
      ball color=black!80},
      ]

      % vertices
      \foreach \i/\o/\l [evaluate=\i as \x using \i*3.5] in
      {
      1/above/\texttt{l}^0_{j,1,\TRUE},
      2/below/\texttt{a}^0_{j,\LEFT},
      3/above/\texttt{l}^0_{j,1,\FALSE},
      4/below/\texttt{b}^0_{j,\LEFT},
      5/below/\texttt{a}^0_{j,\RIGHT},
      6/above/\texttt{l}^0_{j,2,\FALSE,\FST},
      7/below/\texttt{b}^0_{j,\RIGHT},
      8/above/\texttt{l}^0_{j,2,\TRUE},
      9/below/\texttt{c}^0_{j,\LEFT},
      10/above/\texttt{l}^0_{j,2,\FALSE,\SND},
      11/below/\texttt{d}^0_{j,\LEFT},
      12/below/\texttt{c}^0_{j,\RIGHT},
      13/above/\texttt{l}^0_{j,3,\FALSE},
      14/below/\texttt{d}^0_{j,\RIGHT},
      15/above/\texttt{l}^0_{j,3,\TRUE}
      }
      {
      \node [my vertex,label=\o:{\scriptsize $\l$}] (V\i) at (\x,0) {};
      }

      % edges
      \path [edge,color=blue] (V2.north) -- ++ (0,4) --
      ($(V5.north)+(0,4)$) node [pos=0.4,above] {$\COLOR_1$} -- (V5.north);

      \path [edge,color=red] (V4.north) -- ++ (0,5) --
      ($(V7.north)+(0,5)$) node [pos=0.6,above] {$\COLOR_3$} -- (V7.north);

      \path [edge,color=blue] (V9.north) -- ++ (0,5) --
      ($(V12.north)+(0,5)$) node [pos=0.4,above] {$\COLOR_1$} -- (V12.north);

      \path [edge,color=red] (V11.north) -- ++ (0,4) --
      ($(V14.north)+(0,4)$) node [pos=0.6,above] {$\COLOR_3$} -- (V14.north);

      % incoming edges
      \draw [edge,color=blue,path fading=west] (V1) -- node [pos=0.75,anchor=west] {$\COLOR_1$}
      ++(0,-5) -- +(-3,0) node {};

      \draw [edge,color=green,path fading=west] (V3) -- node [pos=0.75,anchor=west] {$\COLOR_2$}
      ++(0,-6) -- +(-10.5,0) node {};

      \draw [edge,color=blue,path fading=west] (V6)  -- node [pos=0.75,anchor=west] {$\COLOR_1$}
      ++(0,-7)  -- +(-21.5,0) node {};

      \draw [edge,color=green,path fading=west] (V8)  -- node [pos=0.75,anchor=west] {$\COLOR_2$}
      ++(0,-8)  -- +(-29,0) node {};

      \draw [edge,color=red,path fading=west] (V10) -- node [pos=0.75,anchor=west] {$\COLOR_3$}
      ++(0,-9) -- +(-36.5,0) node {};

      \draw [edge,color=green,path fading=west] (V13) -- node [pos=0.75,anchor=west] {$\COLOR_2$}
      ++(0,-10) -- +(-47.5,0) node {};

      \draw [edge,color=blue,path fading=west] (V15) -- node [pos=0.75,anchor=west] {$\COLOR_1$}
      ++(0,-11) -- +(-55,0) node {};

      %\node [text width=2.5cm,anchor=west] at (-15.5,-8.5) {\scriptsize Towards the variable gadgets};

    \end{tikzpicture}
    \caption{%
    Clause gadget $\texttt{Cls}^{0}_{j}$ for clause $(\ell_{j,1} \vee \ell_{j,2} \vee \ell_{j,3})$
    in case the literals $\ell_{j,1}$, $\ell_{j,2}$ and $\ell_{j,3}$ evaluate to true, false and true,
    respectivelly, according to the satisfying assignment $f$.
    }% end caption
  \end{subfigure}

  \vspace{.5cm}

  \centering
  \begin{subfigure}[b]{\textwidth}
    \centering
    \begin{tikzpicture}
      [
      scale=.22,,,
      my vertex/.style={basic vertex,
      minimum size=1pt,
      inner sep=2pt,
      ball color=black!80},
      ]

      % vertices
      \foreach \i/\o/\l [evaluate=\i as \x using \i*3.5] in
      {
      1/above/\texttt{l}^0_{j,1,\TRUE},
      2/below/\texttt{a}^0_{j,\LEFT},
      3/above/\texttt{l}^0_{j,1,\FALSE},
      4/below/\texttt{b}^0_{j,\LEFT},
      5/below/\texttt{a}^0_{j,\RIGHT},
      6/above/\texttt{l}^0_{j,2,\FALSE,\FST},
      7/below/\texttt{b}^0_{j,\RIGHT},
      8/above/\texttt{l}^0_{j,2,\TRUE},
      9/below/\texttt{c}^0_{j,\LEFT},
      10/above/\texttt{l}^0_{j,2,\FALSE,\SND},
      11/below/\texttt{d}^0_{j,\LEFT},
      12/below/\texttt{c}^0_{j,\RIGHT},
      13/above/\texttt{l}^0_{j,3,\FALSE},
      14/below/\texttt{d}^0_{j,\RIGHT},
      15/above/\texttt{l}^0_{j,3,\TRUE}
      }
      {
      \node [my vertex,label=\o:{\scriptsize $\l$}] (V\i) at (\x,0) {};
      }

      % edges
      \path [edge,color=red] (V2.north) -- ++ (0,4) --
      ($(V5.north)+(0,4)$) node [pos=0.4,above] {$\COLOR_3$} -- (V5.north);

      \path [edge,color=blue] (V4.north) -- ++ (0,5) --
      ($(V7.north)+(0,5)$) node [pos=0.6,above] {$\COLOR_1$} -- (V7.north);

      \path [edge,color=red] (V9.north) -- ++ (0,5) --
      ($(V12.north)+(0,5)$) node [pos=0.4,above] {$\COLOR_3$} -- (V12.north);

      \path [edge,color=blue] (V11.north) -- ++ (0,4) --
      ($(V14.north)+(0,4)$) node [pos=0.6,above] {$\COLOR_1$} -- (V14.north);

      % incoming edges
      \draw [edge,color=green,path fading=west] (V1) -- node [pos=0.75,anchor=west] {$\COLOR_2$}
      ++(0,-5) -- +(-3,0) node {};

      \draw [edge,color=blue,path fading=west] (V3) -- node [pos=0.75,anchor=west] {$\COLOR_1$}
      ++(0,-6) -- +(-10.5,0) node {};

      \draw [edge,color=red,path fading=west] (V6)  -- node [pos=0.75,anchor=west] {$\COLOR_3$}
      ++(0,-7)  -- +(-21.5,0) node {};

      \draw [edge,color=blue,path fading=west] (V8)  -- node [pos=0.75,anchor=west] {$\COLOR_1$}
      ++(0,-8)  -- +(-29,0) node {};

      \draw [edge,color=green,path fading=west] (V10) -- node [pos=0.75,anchor=west] {$\COLOR_2$}
      ++(0,-9) -- +(-36.5,0) node {};

      \draw [edge,color=green,path fading=west] (V13) -- node [pos=0.75,anchor=west] {$\COLOR_2$}
      ++(0,-10) -- +(-47.5,0) node {};

      \draw [edge,color=blue,path fading=west] (V15) -- node [pos=0.75,anchor=west] {$\COLOR_1$}
      ++(0,-11) -- +(-55,0) node {};

      %\node [text width=2.5cm,anchor=west] at (-15.5,-8.5) {\scriptsize Towards the variable gadgets};

    \end{tikzpicture}
    \caption{%
    Clause gadget $\texttt{Cls}^{0}_{j}$ for clause $(\ell_{j,1} \vee \ell_{j,2} \vee \ell_{j,3})$
    in case the literals $\ell_{j,1}$, $\ell_{j,2}$ and $\ell_{j,3}$ evaluate to false, true and true,
    respectivelly, according to the satisfying assignment $f$.
    }% end caption
  \end{subfigure}

  \vspace{.5cm}

  \begin{subfigure}[b]{\textwidth}
    \centering
    \begin{tikzpicture}
      [
      scale=.22,,,
      my vertex/.style={basic vertex,
      minimum size=1pt,
      inner sep=2pt,
      ball color=black!80},
      ]

      % vertices
      \foreach \i/\o/\l [evaluate=\i as \x using \i*3.5] in
      {
      1/above/\texttt{l}^0_{j,1,\TRUE},
      2/below/\texttt{a}^0_{j,\LEFT},
      3/above/\texttt{l}^0_{j,1,\FALSE},
      4/below/\texttt{b}^0_{j,\LEFT},
      5/below/\texttt{a}^0_{j,\RIGHT},
      6/above/\texttt{l}^0_{j,2,\FALSE,\FST},
      7/below/\texttt{b}^0_{j,\RIGHT},
      8/above/\texttt{l}^0_{j,2,\TRUE},
      9/below/\texttt{c}^0_{j,\LEFT},
      10/above/\texttt{l}^0_{j,2,\FALSE,\SND},
      11/below/\texttt{d}^0_{j,\LEFT},
      12/below/\texttt{c}^0_{j,\RIGHT},
      13/above/\texttt{l}^0_{j,3,\FALSE},
      14/below/\texttt{d}^0_{j,\RIGHT},
      15/above/\texttt{l}^0_{j,3,\TRUE}
      }
      {
      \node [my vertex,label=\o:{\scriptsize $\l$}] (V\i) at (\x,0) {};
      }

      % edges
      \path [edge,color=blue] (V2.north) -- ++ (0,4) --
      ($(V5.north)+(0,4)$) node [pos=0.4,above] {$\COLOR_1$} -- (V5.north);

      \path [edge,color=red] (V4.north) -- ++ (0,5) --
      ($(V7.north)+(0,5)$) node [pos=0.6,above] {$\COLOR_3$} -- (V7.north);

      \path [edge,color=blue] (V9.north) -- ++ (0,5) --
      ($(V12.north)+(0,5)$) node [pos=0.4,above] {$\COLOR_1$} -- (V12.north);

      \path [edge,color=red] (V11.north) -- ++ (0,4) --
      ($(V14.north)+(0,4)$) node [pos=0.6,above] {$\COLOR_3$} -- (V14.north);

      % incoming edges
      \draw [edge,color=blue,path fading=west] (V1) -- node [pos=0.75,anchor=west] {$\COLOR_1$}
      ++(0,-5) -- +(-3,0) node {};

      \draw [edge,color=green,path fading=west] (V3) -- node [pos=0.75,anchor=west] {$\COLOR_2$}
      ++(0,-6) -- +(-10.5,0) node {};

      \draw [edge,color=green,path fading=west] (V6)  -- node [pos=0.75,anchor=west] {$\COLOR_2$}
      ++(0,-7)  -- +(-21.5,0) node {};

      \draw [edge,color=blue,path fading=west] (V8)  -- node [pos=0.75,anchor=west] {$\COLOR_1$}
       ++(0,-8)  -- +(-29,0) node {};

      \draw [edge,color=red,path fading=west] (V10) -- node [pos=0.75,anchor=west] {$\COLOR_3$}
      ++(0,-9) -- +(-36.5,0) node {};

      \draw [edge,color=green,path fading=west] (V13) -- node [pos=0.75,anchor=west] {$\COLOR_2$}
      ++(0,-10) -- +(-47.5,0) node {};

      \draw [edge,color=blue,path fading=west] (V15) -- node [pos=0.75,anchor=west] {$\COLOR_1$}
      ++(0,-11) -- +(-55,0) node {};

      %\node [text width=2.5cm,anchor=west] at (-15.5,-8.5) {\scriptsize Towards the variable gadgets};

    \end{tikzpicture}
    \caption{%
    Clause gadget $\texttt{Cls}^{0}_{j}$ for clause $(\ell_{j,1} \vee \ell_{j,2} \vee \ell_{j,3})$
    in case the literals $\ell_{j,1}$, $\ell_{j,2}$ and $\ell_{j,3}$ evaluate to true, true and true,
    respectivelly, according to the satisfying assignment $f$.
    }% end caption
  \end{subfigure}
  \caption{The general caption.}
\end{figure}



\end{proof}
