\section{Conclusion}
\label{section:Conclusion}

The rationale for introducing \textsc{$k$-Permutation Coloring} stems
from square permutations (\emph{i.e.} those permutations $\pi$ that are
$(\sigma, \sigma)$-colorable for some $\sigma$).
Given a permutation $\pi$, it is \NP-complete to decide
whether there exists some permutations $\sigma$ such that
$\pi \in \sigma \bullet \sigma$ \cite{DBLP:journals/tcs/GiraudoV18}.
How hard becomes the problem if $\sigma$ is part of the input ?
In other words, given permutation $\pi$ and $\sigma$, how hard is the problem
to deciding whether $\sigma$ is a square root
of $\pi$?
