\section{Concluding remarks}
\label{section:Concluding remarks}

The rationale for introducing \textsc{$k$-Merge Permutation} stems
from square permutations (\emph{i.e.} those permutations $\pi^0$ that are the merge 
of two copies of some permutation $\pi^1$).
Given a permutation $\pi^0$, it is \NP-complete to decide
whether there exists some permutations $\pi^1$ such that
$\pi$ is a merge of $\pi^1$ and $\pi^1$ \cite{DBLP:journals/tcs/GiraudoV18}.
The problem is still hard if $\pi^0$ is $(213,231)$-avoiding \cite{DBLP:journals/tcs/GiraudoV18,DBLP:journals/tcs/BulteauV20}.
How hard becomes the problem if $\pi^1$ is part of the input ?
In other words, given permutations $\pi^0$ and $\pi^1$, how hard is the problem
to deciding whether $\pi^0$ is a merge of $\pi^1$ and $\pi^1$?
It is instructive to consider related problems for strings.
Indeed, deciding wether a string is a merge of two copies of some srings 
(\emph{i.e.} does there exist $v$ such that $u \in v \shuffle v$?) is \NP-complete
\cite{DBLP:journals/jcss/BussS14,RIZZI2017}, whereas
deciding wether a string $u$ is a merge of two copies of some given string $v$ 
(\emph{i.e.}, $u \in v \shuffle v$?) is 
polynomial-time solvable \cite{skienna08} (exercice 8.2).

