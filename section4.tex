\section{Merging permutations}
\label{section:Merging permutations}

We are back to merging permutations. 
We show how to transform matching merges to permutation merges.
We need a new definition.
Let $G = (V, E)$ be a linear graph. For any two connected vertices
$v_i, v_j \in V$, $i < j$ or $j < i$,
we define
$$
\RANK(v_i, v_j) = \left|\left\{v_k \in V : \{v_i, v_k\} \in E \text{ and } k \leq j\right\}\right|.
$$
In other words, $\RANK(v_i, v_j) = k+1$ if there exist exactly $k$ vertices
$v_{i_1}, v_{i_2}, \dots, v_{i_k} \in N(v_i)$
with $\max \{i_1, i_2, \dots, i_k\} < j$.

\begin{proposition}
  \label{proposition:3-merge permutation is NP-complete}
  \textsc{$3$-Merge Permutation} is \NP-complete.
\end{proposition}

\begin{proof}
  We reduce from \textsc{$3$-Merge Matching} (Proposition~\ref{proposition:3-Linear Graph Coloring is NP-complete}).
  Let $\mathcal{M}^{0}$, $\mathcal{M}^{1}$, $\mathcal{M}^{2}$, $\mathcal{M}^{3}$ be four matchings. 
  Write $\mathcal{M}^{i} = (V^i, E^i)$,
  $n_i = |V^i|$ and $m_i = |E^i|$ for $0 \leq i \leq 3$.
  We show how to construct permutations $\pi^0$, $\pi^1$, $\pi^2$ and $\pi^3$ such that
  the matching $\mathcal{M}^{0}$ is a merge of the matchings $\mathcal{M}^{1}$, $\mathcal{M}^{2}$ and $\mathcal{M}^{3}$
  if and only if
  the permutation $\pi^0$ is a merge of the permutations $\pi^0$, $\pi^2$ and $\pi^3$.

  Define $N = 8m_0 + 1$.
  We first associate the permutation $\pi^0$ to the matching $\mathcal{M}^{0}$.
  Write $V^0 = \{v^0_i : 1 \leq i \leq n_0\}$ and
  $E^0 = \{e^0_i : 1 \leq i \leq m_0\}$, where the edges are sorted according to their minimum vertex.
  Define the permutations
  \begin{align*}
    \pi(V)
    &=
    \left(\bigoplus_{i=1}^{2m_0} 4i \; (4i-3)\right) [6N_0 + 2m_0]
    \\
    \pi(E)
    &=
    \bigoplus_{j=1}^{m_0} 2 \; f^0(e^0_j) \; g^0(e^0_j) \; 1
    \\
    \pi(\text{Sep}_1)
    &=
    \left(\bigominus_{i=1}^{k} \mathbf{\nearrow}_{N_0}\right) [2n]
    \\
    \pi(\text{Sep}_2)
    &=
    \left(\bigominus_{i=1}^{k} \mathbf{\nearrow}_{N_0}\right) [2n+kN]
    \\
    \intertext{where $f : E^0 \to \mathbb{N}$ and $g : E^0 \to \mathbb{N}$
    are defined as follows}
    \forall e^0_j = (v^0_p, v^0_q) \in E_0,\quad f^0(e^0_j)
    &=
    2m_0 + 2kN_0 + \sum_{j=1}^{p-1}\left(2+d(v^0_j)\right) + \RANK\left(v^0_p, v^0_q\right)
    \\
    \forall e^0_j = (v^0_p, v^0_q) \in E_0,\quad g^0(e^0_j)
    &=
    2m_0 + 2kN_0 + \sum_{j=1}^{q-1}\left(2+d(v^0_j)\right) + \RANK\left(v^0_q, v^0_p\right)\text{.}
    \end{align*}

    In other words,
    $\pi(V)$ is the direct sum of $n_0$ decreasing permutations
    that are all order-isomorphic to $21$,
    $\pi(E)$ is the direct sum of $m_0$ permutations that are all
    order-isomorphic to $2341$ (since $2 < f(e^0_i) < g(e^0_i)$
    for $1 \leq i \leq m_0$), and
    both $\pi(\text{Sep}_1)$ and $\pi(\text{Sep}_2)$ are the skew sums
    of $k$ permutations that are all order isomorphic to $12 \dots k$.
    Finally, we obtain the target permutation $\pi$ by gathering the parts
    $\pi(V)$, $\pi(\text{Sep}_1)$, $\pi(E)$ and $\pi(\text{Sep}_2)$:
    $$
      \pi =
      \pi(V)   \;
      \pi(\text{Sep}_1) \;
      \pi(E)   \;
      \pi(\text{Sep}_2)\text{.}
    $$


  \begin{figure}[htbp!]
  \centering
  \begin{tikzpicture}
  [
    scale=.65,
    box/.style={
      shade,
      blur shadow={shadow blur steps=5,shadow blur extra rounding=1.3pt},
      top color=black!5,
      bottom color=black!50
    },
    gadget/.style={
      box,
      rounded corners
    },
    r/.style={
      box,
      draw,
      ultra thick,
    }
  ]
  \draw [help lines,step=2cm,black!30,fill=black!10] (0,0) grid (8,8);
  \draw [gadget] (0,6) rectangle ++(2,2);
  \node [] at (1,7) {$\pi\left(V\right)$};
  \draw [gadget] (2,2) rectangle ++(2,2);
  \node [] at (3,3) {$\pi\left(\text{Sep}_1\right)$};
  \draw [gadget] (4,0) rectangle ++(2,2);
  \node [] at (5,1) {$\pi\left(E\right)$};
  \draw [gadget] (6,4) rectangle ++(2,2);
  \node [] at (7,5) {$\pi\left(\text{Sep}_2\right)$};
  \draw [gadget] (4,6) rectangle ++(2,2);
  \node [] at (5,7) {$\pi\left(E\right)$};
    % labels
    % x-labels
    \node [label={[text depth=1.75ex,label distance=0.2cm,rotate=-90]right:{$1$}}] at (0,0) {};
    % \node [label={[text depth=1.75ex,label distance=0.2cm,rotate=-90]right:{$2n$}}] at (1.5,0) {};
    \node [label={[text depth=1.75ex,label distance=0.2cm,rotate=-90]right:{$2n+1$}}] at (2,0) {};
    % \node [label={[text depth=1.75ex,label distance=0.2cm,rotate=-90]right:{$2n+kN$}}] at (3.5,0) {};
    \node [label={[text depth=1.75ex,label distance=0.2cm,rotate=-90]right:{$2n+kN+1$}}] at (4,0) {};
    % \node [label={[text depth=1.75ex,label distance=0.2cm,rotate=-90]right:{$2n+kN+4m$}}] at (5.5,0) {};
    \node [label={[text depth=1.75ex,label distance=0.2cm,rotate=-90]right:{$2n+kN+4m+1$}}] at (6,0) {};
    \node [label={[text depth=1.75ex,label distance=0.2cm,rotate=-90]right:{$2n+2kN+4m$}}] at (7.5,0) {};
    % y-labels
    \node [label={[text depth=1.75ex,label distance=0.2cm]left:{$1$}}] at (0,0.1) {};
    % \node [label={[text depth=1.75ex,label distance=0.2cm]left:{$2m2$}}] at (0,1.6) {};
    \node [label={[text depth=1.75ex,label distance=0.2cm]left:{$2m+1$}}] at (0,2.1) {};
    % \node [label={[text depth=1.75ex,label distance=0.2cm]left:{$2m+kN$}}] at (0,3.6) {};
    \node [label={[text depth=1.75ex,label distance=0.2cm]left:{$2m+kN+1$}}] at (0,4.1) {};
    % \node [label={[text depth=1.75ex,label distance=0.2cm]left:{$2m+2kN$}}] at (0,5.6) {};
    \node [label={[text depth=1.75ex,label distance=0.2cm]left:{$2m+2kN+1$}}] at (0,6.1) {};
    \node [label={[text depth=1.75ex,label distance=0.2cm]left:{$2n+2kN+4m$}}] at (0,7.6) {};
  \end{tikzpicture}
  \caption{\label{fig:NP2:bird eye view pi}%
    Bird's-eye view of the permutation $\pi$.
  }% end caption
\end{figure}

  
\begin{figure}
  \centering
  \begin{tikzpicture}
  [
    scale=.65,
    box/.style={
      shade,
      blur shadow={shadow blur steps=5,shadow blur extra rounding=1.3pt},
      top color=black!5,
      bottom color=black!50
    },
    monotone/.style={>=stealth',->,thick,shorten >=.5pt,shorten <=3pt},
  ]
    % separator gadgets
    % monotonic sequences
    \draw [help lines,step=2cm,black!30,fill=black!10] (0,0) grid (8,8);
    % I_1
    \draw [box,rounded corners] (0,6) rectangle ++(2,2);
    \foreach \x/\y in {0.25/6.25,0.5/6.5,1.75/7.75}
      \draw [fill=black] (\x,\y) circle (0.1);
    \draw [thick,loosely dotted,thick,shorten >=1pt,shorten <=1pt] (.75,6.75) -- (1.5,7.5);
    % I_2
    \draw [box,rounded corners] (2,4) rectangle ++(2,2);
    \foreach \x/\y in {2.25/4.25,2.5/4.5,3.75/5.75}
      \draw [fill=black] (\x,\y) circle (0.1);
    \draw [thick,loosely dotted,thick,shorten >=1pt,shorten <=1pt] (2.75,4.75) -- (3.5,5.5);
    % again and angain
    \draw [thick,loosely dotted,thick,shorten >=4pt,shorten <=4pt] (4,4) -- (6,2);
    % I_k
    \draw [box,rounded corners] (6,0) rectangle ++(2,2);
    \foreach \x/\y in {6.25/.25,6.5/.5,7.75/1.75}
      \draw [fill=black] (\x,\y) circle (0.1);
    \draw [thick,loosely dotted,thick,shorten >=1pt,shorten <=1pt] (6.75,.75) -- (7.5,1.5);
    % labels
    % x-labels
    \node [label={[text depth=1.75ex,label distance=0.2cm,rotate=-90]right:{$x+1$}}] at (0,0) {};
    % \node [label={[text depth=1.75ex,label distance=0.2cm,rotate=-90]right:{$x+N_0$}}] at (1.5,0) {};
    \node [label={[text depth=1.75ex,label distance=0.2cm,rotate=-90]right:{$x+N_0+1$}}] at (2,0) {};
    % \node [label={[text depth=1.75ex,label distance=0.2cm,rotate=-90]right:{$x+2N_0$}}] at (3.5,0) {};
    \node [label={[text depth=1.75ex,label distance=0.2cm,rotate=-90]right:{$x+2N_0+1$}}] at (4,0) {};
    % \node [label={[text depth=1.75ex,label distance=0.2cm,rotate=-90]right:{$x+(k-1)N_0$}}] at (5.5,0) {};
    \node [label={[text depth=1.75ex,label distance=0.2cm,rotate=-90]right:{$x+(k-1)N_0+1$}}] at (6,0) {};
    \node [label={[text depth=1.75ex,label distance=0.2cm,rotate=-90]right:{$x+kN_0$}}] at (7.5,0) {};
    % y-labels
    \node [label={[text depth=1.75ex,label distance=0.2cm]left:{$y+1$}}] at (0,0.1) {};
    % \node [label={[text depth=1.75ex,label distance=0.2cm]left:{$y+N_0$}}] at (0,1.6) {};
    \node [label={[text depth=1.75ex,label distance=0.2cm]left:{$y+N_0+1$}}] at (0,2.1) {};
    % \node [label={[text depth=1.75ex,label distance=0.2cm]left:{$y+2N_0$}}] at (0,3.6) {};
    \node [label={[text depth=1.75ex,label distance=0.2cm]left:{$y+2N_0+1$}}] at (0,4.1) {};
    % \node [label={[text depth=1.75ex,label distance=0.2cm]left:{$y+(k-1)N_0$}}] at (0,5.6) {};
    \node [label={[text depth=1.75ex,label distance=0.2cm]left:{$y+(k-1)N_0+1$}}] at (0,6.1) {};
    \node [label={[text depth=1.75ex,label distance=0.2cm]left:{$y+kN_0$}}] at (0,7.6) {};
  \end{tikzpicture}
  \caption{\label{fig:NP2:separators}%
    Schematic representation of $\pi(\text{Sep}_1)$ and $\pi(\text{Sep}_2)$,
    where the offsetis $O=2n$ for
    $\pi(\text{Sep}_1)$ and $O=2n+kN$ for $\pi(\text{Sep}_2)$.
  }% end caption
\end{figure}

  \begin{figure}
  \centering
  \begin{tikzpicture}
  [
    scale=.425,
    box/.style={
      shade,
      blur shadow={shadow blur steps=5,shadow blur extra rounding=1.3pt},
      top color=black!5,
      bottom color=black!50
    },
    gadget/.style={
      box,
      rounded corners
    },
    r/.style={
      box,
      draw,
      ultra thick,
    }
  ]
    % V gadget
    \fill [gadget] (-0.25,9.75) rectangle (7.25,29.25);
    \node at (0.5,27.5) {$\mathcal{V}$};
    % vertex 1
    \path [r] (0,10) -- ++(1,0) -- ++(0,2) -- ++(-1,0) -- cycle;
    \foreach \x/\y in {0.25/11.75,0.75/10.25}
      \draw [fill=black] (\x,\y) circle (0.13);
    \node at (2,11) {$u_1$};
    % again and again
    \draw [thick,loosely dotted,thick,shorten >=4pt,shorten <=4pt] (1,12) -- (2,14);
    % vertex i
    \path [r] (2,14) -- ++(1,0) -- ++(0,4) -- ++(-1,0) -- cycle;
    \draw [thick,step=0.5cm] (2,14) grid (3,18);
    \path [draw,ultra thick] (2,14) -- ++(1,0) -- ++(0,4) -- ++(-1,0) -- cycle;
    \foreach \x/\y in {2.25/17.75,2.75/14.25}
      \draw [fill=black] (\x,\y) circle (0.13);
    \node at (1,16) {$u_i$};
    % again and again
    \draw [thick,loosely dotted,thick,shorten >=4pt,shorten <=4pt] (3,18) -- (4,20);
    % vertex j
    \path [r] (4,20) -- ++(1,0) -- ++(0,5) -- ++(-1,0) -- cycle;
    \draw [thick,step=0.5cm] (4,20) grid (5,25);
    \path [draw,ultra thick] (4,20) -- ++(1,0) -- ++(0,5) -- ++(-1,0) -- cycle;
    \foreach \x/\y in {4.25/24.75,4.75/20.25}
      \draw [fill=black] (\x,\y) circle (0.13);
    \node at (3,22.5) {$u_j$};
    % again and again
    \draw [thick,loosely dotted,thick,shorten >=4pt,shorten <=4pt] (5,25) -- (6,27);
    % vertex n
    \path [r] (6,27) -- ++(1,0) -- ++(0,2) -- ++(-1,0) -- cycle;
    \foreach \x/\y in {6.25/28.75,6.75/27.25}
      \draw [fill=black] (\x,\y) circle (0.13);
    \node at (5,28) {$u_n$};

    % E gadget
    \fill [gadget] (8.75,2.75) rectangle (19.25,8.25);
    \node at (9.5,7.5) {$\mathcal{E}$};
    % first edge
    \path [r] (9,3) -- ++(2,0) -- ++(0,1) -- ++(-2,0) -- cycle;
    \draw [thick,step=0.5cm] (9,3) grid (11,4);
    \path [draw,ultra thick] (9,3) -- ++(2,0) -- ++(0,1) -- ++(-2,0) -- cycle;
    \foreach \x/\y in {9.25/3.75,10.75/3.25}
      \draw [fill=black] (\x,\y) circle (0.13);
    \node at (10,4.75) {$e_1$};
    % again and again
    \draw [thick,loosely dotted,thick,shorten >=4pt,shorten <=4pt] (11,4) -- (13,5);
    % edge (i,j))
    \path [r] (13,5) -- ++(2,0) -- ++(0,1) -- ++(-2,0) -- cycle;
    \draw [thick,step=0.5cm] (13,5) grid (15,6);
    \path [draw,ultra thick] (13,5) -- ++(2,0) -- ++(0,1) -- ++(-2,0) -- cycle;
    \foreach \x/\y in {13.25/5.75,14.75/5.25}
      \draw [fill=black] (\x,\y) circle (0.13);
    \node at (14,4.25) {$e_\ell = \{i,j\}$};
    % again and again
    \draw [thick,loosely dotted,thick,shorten >=4pt,shorten <=4pt] (15,6) -- (17,7);
    % last edge
    \path [r] (17,7) -- ++(2,0) -- ++(0,1) -- ++(-2,0) -- cycle;
    \draw [thick,step=0.5cm] (17,7) grid (19,8);
    \path [draw,ultra thick] (17,7) -- ++(2,0) -- ++(0,1) -- ++(-2,0) -- cycle;
    \foreach \x/\y in {17.25/7.75,18.75/7.25}
      \draw [fill=black] (\x,\y) circle (0.13);
    \node at (18,6.25) {$e_m$};

    % Connection gadget
    \fill [gadget] (8.75,9.75) rectangle (19.25,29.25);
    \node at (9.5,27.5) {$\mathcal{G}$};
    \fill [r] (13-0.001,14-0.001) rectangle (15,18);
    \draw [thick,step=0.5cm,black] (13-0.001,14-0.001) grid (15,18);
    \draw [fill=black] (13.75,15.25) circle (0.13);
    \fill [r] (13-0.001,20-0.001) rectangle (15,25);
    \draw [thick,step=0.5cm,black] (13-0.001,20-0.001) grid (15,25);
    \draw [fill=black] (14.25,23.25) circle (0.13);

    %  alignments
    % vertical
    \draw [black!75,dash pattern={on 4pt off 2pt},shorten >=1pt,shorten <=1pt] (13,6) -- (13,14);
    \draw [black!75,dash pattern={on 4pt off 2pt},shorten >=1pt,shorten <=1pt] (13,18) -- (13,20);
    \draw [black!75,dash pattern={on 4pt off 2pt},shorten >=1pt,shorten <=1pt] (15,6) -- (15,14);
    \draw [black!75,dash pattern={on 4pt off 2pt},shorten >=1pt,shorten <=1pt] (15,18) -- (15,20);
    %
    \draw [black!75,dash pattern={on 4pt off 2pt},shorten >=1pt,shorten <=1pt] (13.5,6) -- (13.5,14);
    \draw [black!75,dash pattern={on 4pt off 2pt},shorten >=1pt,shorten <=1pt] (13.5,18) -- (13.5,20);
    \draw [black!75,dash pattern={on 4pt off 2pt},shorten >=1pt,shorten <=1pt] (14.5,6) -- (14.5,14);
    \draw [black!75,dash pattern={on 4pt off 2pt},shorten >=1pt,shorten <=1pt] (14.5,18) -- (14.5,20);
    % horizontal
    \draw [black!75,dash pattern={on 4pt off 2pt},shorten >=1pt,shorten <=1pt] (3,14) -- (13,14);
    \draw [black!75,dash pattern={on 4pt off 2pt},shorten >=1pt,shorten <=1pt] (3,14.5) -- (13,14.5);
    \draw [black!75,dash pattern={on 4pt off 2pt},shorten >=1pt,shorten <=1pt] (3,17.5) -- (13,17.5);
    \draw [black!75,dash pattern={on 4pt off 2pt},shorten >=1pt,shorten <=1pt] (3,18) -- (13,18);
    \draw [black,line width=1pt,<->,>=latex'] (4,14.5) -- (4,17.5);
    \node [rotate=-90] at (4.75,16) {$d(u_i)$};
    \draw [black!75,dash pattern={on 4pt off 2pt},shorten >=1pt,shorten <=1pt] (5,20) -- (13,20);
    \draw [black!75,dash pattern={on 4pt off 2pt},shorten >=1pt,shorten <=1pt] (5,20.5) -- (13,20.5);
    \draw [black!75,dash pattern={on 4pt off 2pt},shorten >=1pt,shorten <=1pt] (5,24.5) -- (13,24.5);
    \draw [black!75,dash pattern={on 4pt off 2pt},shorten >=1pt,shorten <=1pt] (5,25) -- (13,25);
    \draw [black,line width=1pt,<->,>=latex'] (6,20.5) -- (6,24.5);
    \node [rotate=-90] at (7.25,22) {$d(u_j)$};

    % labels
    % vertex gadget
    % x-labels
    \node [anchor=east,rotate=-90] at (0.25,29.25) {$1$};
    \node [anchor=east,rotate=-90] at (2.25,29.25) {$1 + \sum_{x=1}^{i-1}2d(u_x)$};
    \node [anchor=east,rotate=-90] at (4.25,29.25) {$1 + \sum_{x=1}^{i-1}2d(u_x)$};
    \node [anchor=east,rotate=-90] at (6.25,29.25) {$2n-1$};
    % y-labels
    \node [anchor=east] at (-0.5,10.25) {$2m+2kN+1$};
    \node [anchor=east] at (-0.5,28.75) {$2n+4m+2kN$};
    % edge gadget
    % x-labels
    \node [rotate=-90,anchor=west] at (9.25,2.5) {$2n+kN+1$};
    \node [rotate=-90,anchor=west] at (13.25,2.5) {$2n+kN+4(\ell-1)+1$};
    \node [rotate=-90,anchor=west] at (17.25,2.5) {$2n+kN+4(m-1)+1$};
    % y-labels
    \node [anchor=east] at (8.75,3.25) {$1$};
    \node [anchor=east] at (8.75,5.25) {$2(\ell-1)+1)$};
    \node [anchor=east] at (8.75,7.75) {$2m$};
    % connection labels
    % y-labels
    \node [anchor=east,rotate=-90] at (17,10.5) {$2m+2kN+\sum_{x=1}^{i-1}(d(u_x)+2)+1$};
    \draw [black,line width=1pt,->,>=latex',rounded corners=1.5pt]
      (17,10.5) -- (17,10) -- (16,10) -- (16,14.25) -- (15,14.25);
    \node [anchor=west,rotate=-90] at (18.5,28) {$2m+2kN+\sum_{x=1}^{j-1}(d(u_x)+2)+1$};
    \draw [black,line width=1pt,->,>=latex',rounded corners=1.5pt]
      (18.5,28) -- (18.5,28.5) -- (16,28.5) -- (16,20.25) -- (15,20.25);
  \end{tikzpicture}
  \caption{\label{fig:NP2:connection zoom}%
  $\RANK(u_i, u_j) = 2$ and $\RANK(u_j, u_i) = 6 = d(u_j)$.
  }% end caption
\end{figure}


  A simplified representation of the permutation $\pi$ is given in
  Figure~\ref{fig:NP2:bird eye view pi}.
  A schematic representation of $\pi(\text{Sep}_1)$ and $\pi(\text{Sep}_2)$
  is given in Figure~\ref{fig:NP2:separators}, where the offset is $O=2n$ for
  $\pi(\text{Sep}_1)$ and $O=2n+kN$ for $\pi(\text{Sep}_2)$.
  A zoom of $\pi(E)$ is given in Figure~\ref{fig:NP2:connection zoom}.

  For example, referring Figure~\ref{fig:3-Linear-Graph Coloring} ($k=3$), we
  obtain from the linear graph $G^0$:
  $N_0 = 26 + 44 + 1 = 71$, $2m0+2kN_0= 22 + 426 = 448$, and hence
  $\pi(V) = \left(3 1 \oplus 3 1 \oplus 4 1 \oplus 4 1 \oplus 4 1 \oplus
  4 1 \oplus 4 1 \oplus 4 1 \oplus 4 1 \oplus 3 1 \oplus 3 1 \oplus 4 1 \oplus 3 1\right) [448]$,
  $\pi(E) = 2\;449\;479\;1 \oplus \dots$

  We now turn to associating the permutation $\sigma_i$, $1 \leq i \leq k$,
  to the linear graph $G^i = (V^i, E^i)$.
  The construction, as a whole, follows the same line as of $\pi$,
  the main difference being the \emph{separators}
  $\sigma_i\left(\text{Sep}_1\right)$ and
  $\sigma_i\left(\text{Sep}_2\right)$ that are both simplified.
  Define
  \begin{align*}
    \sigma_i(V)
    &=
    \left(\bigoplus_{j=1}^{n_i} (d(v^i_j)+2) \; 1\right) [2m_i+2N_i]
    \\
    \sigma_i(E)
    &=
    \bigoplus_{j=1}^{m_i} 2 \; f^i(e^0_j) \; g^i(e^0_j) \; 1
    \\
    \sigma_i\left(\text{Sep}_1\right)
    &=
    \mathbf{\nearrow}_{N_0} [2n_i]
    \\
    \sigma_i\left(\text{Sep}_2\right)
    &=
    \mathbf{\nearrow}_{N_0} [2n_i+N]
    \\
    \intertext{where $f^i : E^i \to \mathbb{N}$ and $g^i : E^i \to \mathbb{N}$
    are defined as follows}
    \forall e^i_j = (v^i_p, v^i_q) \in E_i,\quad f^i(e^0_j)
    &=
    2m_0 + 2kN_0 + \sum_{j=1}^{p-1}\left(2+d(v^0_j)\right) + \RANK\left(v^0_p, v^0_q\right)
    \\
    \forall e^i_j = (v^i_p, v^0_q) \in E_i,\quad g^i(e^0_j)
    &=
    2m_0 + 2kN_0 + \sum_{j=1}^{q-1}\left(2+d(v^0_j)\right) + \RANK\left(v^0_q, v^0_p\right)\text{.}
    \end{align*}
  %
  %   \intertext{where (assuming edge $e_{i,\ell} = (u_i, u_j)$)}
  %   x^1_{i,\ell}
  %   &=
  %   2m + 2kN + \sum_{\ell'=1}^{i-1}\left(2+d(u_{\ell'})\right) + \RANK\left(u_i, u_j\right)
  %   \\
  %   x^2_{i,\ell}
  %   &=
  %   2m + 2kN + \sum_{\ell'=1}^{j-1}\left(2+d(u_{\ell'})\right) + \RANK\left(u_j, u_i\right),
  %   \\
  %   \intertext{and set}
  %   \sigma_i
  %   &=
  %   \sigma_i(V)   \;
  %   \sigma_i\left(\text{Sep}_1\right) \;
  %   \sigma_i\left(E\right)   \;
  %   \sigma_i\left(\text{Sep}_1\right)\text{.}
  % \end{align*}

  \begin{figure}[htbp!]
  \centering
  \begin{tikzpicture}
  [
    scale=.65,
    box/.style={
      shade,
      blur shadow={shadow blur steps=5,shadow blur extra rounding=1.3pt},
      top color=black!5,
      bottom color=black!50
    },
    gadget/.style={
      box,
      rounded corners
    },
    r/.style={
      box,
      draw,
      ultra thick,
    }
  ]
    \draw [help lines,step=2cm,black!30,fill=black!10] (0,0) grid (8,8);
    \draw [gadget] (0,6) rectangle ++(2,2);
    \node [] at (1,7) {$\sigma_i\left(V\right)$};
    \draw [gadget] (2,2) rectangle ++(2,2);
    \node [] at (3,3) {$\sigma_i\left(\text{Sep}_1\right)$};
    \draw [gadget] (4,0) rectangle ++(2,2);
    \node [] at (5,1) {$\sigma_i\left(E\right)$};
    \draw [gadget] (6,4) rectangle ++(2,2);
    \node [] at (7,5) {$\sigma_i\left(\text{Sep}_2\right)$};
    \draw [gadget] (4,6) rectangle ++(2,2);
    \node [] at (5,7) {$\sigma_i\left(E\right)$};
    % labels
    % x-labels
    \node [label={[text depth=1.75ex,label distance=0.2cm,rotate=-90]right:{$1$}}] at (0,0) {};
    % \node [label={[text depth=1.75ex,label distance=0.2cm,rotate=-90]right:{$2n$}}] at (1.5,0) {};
    \node [label={[text depth=1.75ex,label distance=0.2cm,rotate=-90]right:{$2n_i+1$}}] at (2,0) {};
    % \node [label={[text depth=1.75ex,label distance=0.2cm,rotate=-90]right:{$2n+kN$}}] at (3.5,0) {};
    \node [label={[text depth=1.75ex,label distance=0.2cm,rotate=-90]right:{$2n_i+N+1$}}] at (4,0) {};
    % \node [label={[text depth=1.75ex,label distance=0.2cm,rotate=-90]right:{$2n+kN+4m$}}] at (5.5,0) {};
    \node [label={[text depth=1.75ex,label distance=0.2cm,rotate=-90]right:{$2n_i+N+4m_i+1$}}] at (6,0) {};
    \node [label={[text depth=1.75ex,label distance=0.2cm,rotate=-90]right:{$2n_i+2N+4m_i$}}] at (7.5,0) {};
    % y-labels
    \node [label={[text depth=1.75ex,label distance=0.2cm]left:{$1$}}] at (0,0.1) {};
    % \node [label={[text depth=1.75ex,label distance=0.2cm]left:{$2m2$}}] at (0,1.6) {};
    \node [label={[text depth=1.75ex,label distance=0.2cm]left:{$2m_i+1$}}] at (0,2.1) {};
    % \node [label={[text depth=1.75ex,label distance=0.2cm]left:{$2m+kN$}}] at (0,3.6) {};
    \node [label={[text depth=1.75ex,label distance=0.2cm]left:{$2m_i+N+1$}}] at (0,4.1) {};
    % \node [label={[text depth=1.75ex,label distance=0.2cm]left:{$2m+2kN$}}] at (0,5.6) {};
    \node [label={[text depth=1.75ex,label distance=0.2cm]left:{$2m_i+2N+1$}}] at (0,6.1) {};
    \node [label={[text depth=1.75ex,label distance=0.2cm]left:{$2n_i+2N+4m_i$}}] at (0,7.6) {};
  \end{tikzpicture}
  \caption{\label{fig:NP2:bird eye view sigma_i}%
    Bird's-eye view of the permutation $\sigma_i$.
  }% end caption
\end{figure}


  A schematic representation of the permutation $\sigma_i$ is given in
  Figure~\ref{fig:NP2:bird eye view sigma_i}.

  Clearly our construction can be carried on in polynomial time.
  Indeed, we have
  $|\pi| = 2n + 4m + 2kN = 2n + 4m + 2k(2n+4m+1)$
  and
  $|\sigma_i| = 2n_i + 4m_i + 2N = 2n_i + 4m_i + 2(2n+4m+1)$,
  $1 \leq i \leq k$.
  We claim that the linear graph $G^0$ is
  $(G^1, G^2, \dots, G^k)$-colorable
  if and only if the permutation
  $\pi$ is $(\sigma_1, \sigma_2, \dots, \sigma_k)$-colorable.

  Suppose first that $G^0$ is $(G^1, G^2, \dots, G^k)$-colorable.
  Therefore, there exists a $k$-coloring such that
  every connected component of $G^0$ is monochromatic and,
  for every $1 \leq i \leq k$, colour $i$ induces a linear graph that is
  isomorphic to $G^i$.
  For every $1 \leq i \leq k$,
  let $v_{i_1}< v_{i_2} < \dots < v_{i_{n_i}}$ be the vertices of $G^0$ with color $i$.
  Let us now define a $k$-colouring $\varphi: [|\pi|] \to [k]$ of $\pi$ as follows.
  \begin{itemize}
    \item \textbf{Gadget $\pi(V)$}.
    For every $1 \leq j \leq n_i$,
    $\varphi(2i_j-2) = \varphi(2i_j-1) = i$
    \item \textbf{Gadget $\pi(\text{Sep}_1)$}.
    Colour the $i$-th increasing sequence $\nearrow_{N_0}$ with colour $i$.
    More formally,
    for every $2n + N_0(i-1) + 1 \leq j \leq 2n + iN_0$,
    $\varphi(j) = i$.
    \item \textbf{Gadget $\pi(E)$}.
    Let $e_{i_1} < e_{i_2} < \ldots < ae{i_{m_i}}$ be the edges of $G^0$ that connect
    vertices coloured with colour $i$.
    For every $1 \leq j \leq m_i$,
    $\varphi(2n + kN_0 + 4(i_j-1) + 1) = \varphi(2n + kN_0 + 4(i_j-1) + 2) =
    \varphi(2n + kN_0 + 4(i_j-1) + 3) = \varphi(2n + kN_0 + 4(i_j-1) + 4) = i$.
    \item \textbf{Gadget $\pi(\text{Sep}_2)$}.
    Colour the $i$-th increasing sequence $\nearrow_{N_0}$ with colour $i$.
    More formally,
    for every $2n + kN_0 + 4m + N_0(i-1) + 1 \leq j \leq 2n + kN_0 + 4m + iN_0$,
    $\varphi(j) = i$.
  \end{itemize}
  The reader is invited to check that, for every $1 \leq i \leq k$,
  the $i$-coloured pattern of $\pi$ is order-isomorphic to $\sigma_i$.

  Conversely, suppose that $\pi$ is $(\sigma_1, \sigma_2, \dots, \sigma_k)$-colorable.
  Therefore, there exists a $k$-coloring $\varphi: [2n_0 + 4m_0 + 2kN_0] \to [k]$
  such that, for every $1 \leq i \leq k$,
  the $i$-coloured pattern of $\pi$ is order-isomorphic to $\sigma_i$.
  Let us focus on $\sigma_i$ for some $1 \leq i \leq k$.
  Recall that
  $\sigma_i =
  \sigma_i(V)   \;
  \sigma_i\left(\text{Sep}_1\right) \;
  \sigma_i\left(E\right)   \;
  \sigma_i\left(\text{Sep}_1\right)$,
  where both
  $\sigma_i\left(\text{Sep}_1\right)$ and $\sigma_i\left(\text{Sep}_2\right)$
  are increasing sequence of length $N_0$.
  We now oberve that
  $N_0 > |\pi(V)| + |\pi(E)|$.
  Then it follows that
  (i) at least one element of $\pi(\text{Sep}_1)$ is coloured with
  colour $i$,
  and
  (ii) at least of element of $\pi(\text{Sep}_2)$ is coloured with
  colour $i$.
  But
  $\pi(\text{Sep}_1) \; \pi(\text{Sep}_2)$ and
  $\sigma_i\left(\text{Sep}_1\right) \; \sigma_i\left(\text{Sep}_2\right)$
  are both order-isomorphic to
  $\left(\bigominus_{\ell=1}^{k} \nearrow_N\right) \oplus \left(\bigominus_{\ell=1}^{k} \nearrow_N\right)$,
  and hence
  (i) $N$ vertices of $\pi(\text{Sep}_1)$ are coloured with
  colour $i$,
  and
  (ii) $N$ vertices of $\pi(\text{Sep}_1)$ are coloured with
  colour $i$.
  Therefore,
  (i) $|\sigma_i(V)| = 2n_i$ elements of
  $\pi(V)$ are coloured with colour $i$ and the induced
  $i$-coloured pattern is order-isomorphic to $\sigma_i(V)$,
  and
  (ii) $|\sigma_i\left(E\right)| = 4m_i$ elements of
  $\pi(E)$ are coloured with colour $i$ and the induced
  $i$-coloured pattern is order-isomorphic to $\sigma_i\left(E\right)$.
  But $\pi(V)$ is order-isomorphic to $\bigoplus_{\ell=1}^{n} 21$
  and
  $\sigma_i(V)$ is order-isomorphic to $\bigoplus_{\ell=1}^{n_i} 21$.
  Therefore, $n_i$ patterns $21$ are coloured with colour $i$ thereby identifying
  $n_i$ vertices of $G$.
  Finally,
  combining
  $\pi(E) = \bigoplus_{\ell=1}^{m} 2 \; x_\ell^1 \; x_\ell^2 \; 1$ and
  $\sigma_i\left(E\right) = \bigoplus_{\ell=1}^{m_i} 2 \; x_{i,\ell^1} \; x_{i,\ell^2} \; 1$,
  together with the fact that $x_{i,\ell^1}$ and $x_{i,\ell^2}$ have to be sandwiched in
  $i$-coloured pattern $21$ of $\pi(V)$,
  we conclude that $m_i$ consecutive patterns $2 \; x_{i,\ell^1} \; x_{i,\ell^2} \; 1$
  of $\pi(E)$ are coloured with colour $i$.
  Hence, \todo{Ok I'm lazy (or tired), this is not formal!}referring to
  Figure~\ref{fig:NP2:connection zoom} and considering every $1 \leq i \leq k$,
  we conclude that $G$ can be splitted by linear graphs $H_1, H_2, \ldots, H_k$.
  \qed
\end{proof}

\begin{corollary}
  \label{corollary:5-permutation coloring is NP-complete}
  \textsc{$3$-permutation coloring} is \NPC.
\end{corollary}

\begin{proof}
  Combine Proposition~\ref{proposition:3-Linear Graph Coloring}
  with Proposition~\ref{proposition:k-linear-graph coloring < k-permutation coloring}.
  \qed
\end{proof}

It is worth noticing that, according to
Proposition~\ref{proposition:k-linear-graph coloring < k-permutation coloring},
any improvement on Proposition~\ref{proposition:5-linear-graph coloring is NP-complete}
would immediately propagate to \textsc{$k$-permutation coloring}.
