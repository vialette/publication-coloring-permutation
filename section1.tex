\section{Introduction}
\label{section:Introduction}

Given permutations $\pi$ and $\sigma_1, \sigma_2, \ldots, \sigma_k$,
the permutation $\pi$ is said to be
\emph{$(\sigma_1, \sigma_2, \ldots, \sigma_k)$-colorable},
denoted $\pi \in \sigma_1 \bullet \sigma_2 \bullet \dots \bullet \sigma_k$,
if there exists a $k$-coloring of $\pi$ (\emph{i.e.}, assign one color
among $k$ available colors to each element of $\pi$) such that the
pattern induced by each color $i$, $1 \leq i \leq k$, is order-isomorphic
to $\sigma_i$.
For example, $\pi = 192854367$ is 
$({\color{red}12}, {\color{teal}312}, {\color{blue}4213})$-colorable
as shown by 
$\pi = 
\raisebox{2pt}{{\color{red}1}}
\raisebox{0pt}{{\color{teal}92}}
\raisebox{-2pt}{{\color{blue}8}}
\raisebox{2pt}{{\color{red}5}}
\raisebox{-2pt}{{\color{blue}43}}
\raisebox{0pt}{{\color{teal}6}}
\raisebox{-2pt}{{\color{blue}7}}$,
${\color{red}15} \simeq 12$,
${\color{teal}926} \simeq 312$ and
${\color{blue}8437} \simeq 4213$.
Given permutations $\pi$ and $\sigma_1, \sigma_2, \ldots, \sigma_k$,
the \textsc{$k$-Permutation Coloring} problem is to decide whether $\pi$
is $(\sigma_1, \sigma_2, \ldots, \sigma_k)$ -colorable.

Given a permutation $\pi$, deciding whether $\pi$ is the shuffle of 
two order-isomorphic patterns is \NP-complete \cite{DBLP:journals/tcs/GiraudoV18}.
In other words, given a permutation $\pi$, deciding whether 
there exists a permutations $\sigma$ such that $\pi$ is 
$(\sigma, \sigma)$-colorable is \NP-complete,
even if $\pi$ is $(213, 231)$-avoiding.

The main question is: 
given permutations $\pi$ and $\sigma$, is $\pi$ $(\sigma, \sigma)$-colorable?