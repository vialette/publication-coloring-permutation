\section{Introduction}
\label{section:Introduction}

Given permutations $\pi$ and $\sigma_1, \sigma_2, \ldots, \sigma_k$,
the permutation $\pi$ is said to be
\emph{$(\sigma_1, \sigma_2, \ldots, \sigma_k)$-colorable},
denoted $\pi \in \sigma_1 \bullet \sigma_2 \bullet \dots \bullet \sigma_k$,
if there exists a $k$-coloring of $\pi$ (\emph{i.e.}, assign one color
among $k$ available colors to each element of $\pi$) such that the
pattern induced by each color $i$, $1 \leq i \leq k$, is order-isomorphic
to $\sigma_i$.
For example, $\pi = 192854367$ is $(12, 312, 4213)$-colorable
(\emph{i.e.}, $\pi \in 12 \bullet 312 \bullet 4213$) since
the patterns $15$, $926$ and $8537$ are order-isomorphic to
$12$, $312$ and $4213$, respectivelly.

The \textsc{$k$-Permutation Coloring} problem is defined as follows:
Given permutations $\pi$ and $\sigma_i$, $1 \leq i \leq k$, with
$|\pi| = \sum_{i=1}^k |\sigma_i|$,
decide the question
$\pi \in \sigma_1 \bullet \sigma_2 \bullet \dots \bullet \sigma_k$.
